\documentclass[parskip=full]{scrartcl}
\usepackage[top=2.54cm, bottom=2.54cm, left=2.54cm, right=2.54cm]{geometry}
\usepackage[utf8]{inputenc} % use utf8 file encoding for TeX sources
\usepackage[T1]{fontenc}    % avoid garbled Unicode text in pdf
\usepackage[ngerman]{babel}  % german hyphenation, quotes, etc
\usepackage{hyperref}       % hyperlinks in document
\usepackage{glossaries}     % glossary 
\usepackage{enumerate}      % advanced enumeration
\usepackage[shortlabels]{enumitem}
\usepackage[dvipsnames]{xcolor}
\usepackage{graphicx}
\usepackage{caption}        % add captions
\usepackage{adjustbox}      % add adjustmentbox
\usepackage{csquotes}

% Set footer and header bar
\usepackage[headsepline, footsepline]{scrlayer-scrpage}
\addtokomafont{headsepline}{\color{BlueViolet}}
\addtokomafont{footsepline}{\color{BlueViolet}}
\KOMAoptions{headsepline=1.25pt:\textwidth}
\KOMAoptions{footsepline=1.25pt:\textwidth}
\clearpairofpagestyles
\rofoot{\thepage}
\ihead{Write your own Android App: SpiceSquad}

% set the hyperlink style in the document
\hypersetup{
    pdftitle={PSE: Entwurfsheft},
    pdfborderstyle={/S/U/W 1},
    colorlinks,
    linkcolor={black!50!black},
    citecolor={blue!50!black},
    urlcolor={blue!80!black}
}

% sets right quotation for "
\MakeOuterQuote{"}

%change figure numbering
\renewcommand{\thefigure}{\thesection.\arabic{figure}}

%add command to hide subsections from toc
\newcommand{\changelocaltocdepth}[1]{%
  \addtocontents{toc}{\protect\setcounter{tocdepth}{#1}}%
  \setcounter{tocdepth}{#1}%
}

\newcommand{\enablesubsectionnumbering}[1]{
    \renewcommand{\thesubsection}{$\langle$#1\arabic{subsection}0$\rangle$}
    \changelocaltocdepth{1} 
}

\newcommand{\resetsubsectionnumbering}{
    \renewcommand{\thesubsection}{\arabic{section}.\arabic{subsection}}
    \changelocaltocdepth{3} 
}

% glossary
\makeglossaries
\renewcommand{\glossarysection}[2][]{}

\begin{document}
%Glossary entry
\newglossaryentry{squad}{
    name=Squad,
    description={Cooler Begriff für Gruppe}
}

\newglossaryentry{Admin}{
    name=Admin,
    description={Rolle, die innerhalb einer Gruppe eingenommen werden kann. Admins können Nutzer kicken, bannen, zu Admins befördern sowie Rezepte für Gruppenmitglieder unsichtbar machen. Der Ersteller einer Gruppe ist automatisch ein Admin. Admins können von anderen Admins zu normalen Nutzern degradiert werden.}
}

\newglossaryentry{privat}{
    name=privat,
    description={Ein privates Rezept ist nur für den Nutzer, der es erstellt hat, sichtbar.}
}

\newglossaryentry{Rezept}{
    name=Rezept,
    description={Ein Rezept besteht aus einer Liste an Zutaten mit Mengenangaben sowie einer Zubereitungsanweisung in Fließtext. Zusätzlich können Angaben zum Zeitaufwand und zur Schwierigkeit (Hinzufügedatum?) enthalten sein.}
}

\begin{titlepage}
    \begin{center}
        \begin{Huge}
            {\textbf{Write your own Android App: SpiceSquad}}
        \end{Huge}
        \vspace{12px}

        Praxis der Softwareentwicklung (PSE)\\
        Sommersemester 2023\\
        \vspace{150px}

        \begin{Huge}
            {\textbf{Entwurfsheft}}
        \end{Huge}
        \vspace{12px}

        \textbf{Auftraggeber}\\
        Karlsruher Institut für Technologie\\
        KASTEL — Institut für Informationssicherheit und Verlässlichkeit\\
        \vspace{330px}

        \textbf{Auftragnehmer}\\
        Karlsruher Intellektuelle\\
        Henri Becker, Konrad Knappe, Lukas Schwarz, Raphael Zipperer\\
    \end{center}
\end{titlepage}

\tableofcontents
\newpage

\section*{Gender-Hinweis}
Zur besseren Lesbarkeit wird in diesem Entwurfsheft das generische Maskulinum verwendet.
Die in diesem Heft verwendeten Personenbezeichnungen beziehen sich – sofern nicht anders kenntlich gemacht – auf alle Geschlechter.


\section{Presentation layer}
    %\includegraphics[width = 180mm]{entwurfsheft/images/Presentation-layer/Presentation-Layer.png}
    \newpage
    \subsection{Recipe Creation Page}
        Die Klassen beinhaltet alle nötigen Funktionen und Attribute um die Rezept-Erstell-Ansicht darzustellen.\newline
        \textbf{Attributes}
        \begin{itemize}
            \item image: Hochgeladenes Bild welches den anderen Nutzern in der Rezeptansicht angezeigt werden soll.
            \item title: Die eingegebene Zeichenkette für den Titel des Rezepts.
            \item isVegetarian: Der vom Nutzer angegebene Boolean ob das \gls{labels} Vegetarisch dem Leser angezeigt werden soll.
            \item isVegan: Der vom Nutzer angegebene Boolean ob das \gls{labels} Vegan dem Leser angezeigt werden soll.
            \item isGlutenFree: Der vom Nutzer angegebene Boolean ob das \gls{labels} Glutenfrei dem Leser angezeigt werden soll.
            \item isHalal: Der vom Nutzer angegebene Boolean ob das \gls{labels} Halal dem Leser angezeigt werden soll.
            \item isKosher: Der vom Nutzer angegebene Boolean ob das \gls{labels} Koscher dem Leser angezeigt werden soll.
            \item duration: Die vom Nutzer eingegebene Ganzzahl, welche die Zubereitunszeit in Minuten angibt.
            \item dificullty: Die vom Nutzer ausgewählte \gls{schwierigkeit} des Rezepts
            \item ingerdients: Eine Liste an Ingerdients, welche in der AddIngredientPage erstellt wurde:
            \item instructions: Die vom Nutzer angegeben Zeichenkette für die Beschreibung des Rezepts.
        \end{itemize}
        
        \textbf{Methods}
        \begin{itemize}
            \item addIngredient(): Die Page AddIngredientPage wird aufgerufen.
            \item recipeSave(): Die Methode ruft, abhängig davon ob das Rezept neu erstellt wird oder nur verändert wurde, entweder recipeCreation() oder recipeChange() auf.
        \end{itemize}

        \begin{figure}[htp]
            \begin{minipage}
                [t]{0.49\textwidth}
                \centering
                \includegraphics[height=80mm]{pflichtenheft/images/benutzeroberfläche/RegisterView.jpg}
                \caption{Registrierungsansicht}
            \end{minipage}
            \begin{minipage}
                [t]{0.49\textwidth}
                \centering
                \includegraphics[height = 50mm]{entwurfsheft/images/Presentation-layer/Recipe Creation Page.png}
                \caption{RecipeCreationPage}
            \end{minipage}
        \end{figure}

 
\newpage
\section{Glossar}
\printglossary[style=altlist]
\end{document}