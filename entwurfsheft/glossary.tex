\newglossaryentry{label}{
    name=Label,
    plural=Labels,
    description={Rezepte können mit bezeichnenden Stichwörtern, sogenannten Labels, versehen werden. Dies ermöglicht das Filtern von Rezepten nach bestimmten Eigenschaften (z.B. vegetarisch, glutenfrei, halal)}
}

\newglossaryentry{schwierigkeit}{
    name=Schwierigkeit,
    description={Optionale Angabe zur abgeschätzten Schwierigkeit eines Rezeptes. Es gibt die Stufen "einfach", "mittel" und "schwer". Wird beim Erstellen des Rezepts keine Angabe zur Schwierigkeit gemacht, wird diese automatisch auf "mittel" gesetzt}
}

\newglossaryentry{sichtbarkeit}{
    name=Sichtbarkeit,
    description={Rezepte können entweder privat oder öffentlich sein. Private Rezepte können nur vom Autor eingesehen werden. Öffentliche Rezepte können von allen Mitgliedern von Gruppen, in denen der Autor sich befindet und in welchen das Rezept nicht ausgeblendet ist, eingesehen werden}   
}

\newglossaryentry{blob}{
    name=BLOB,
    description={Binary Large Object. Ein Datentyp, der eine große Menge an binären Daten speichern kann}
}

\newglossaryentry{uuid}{
    name=UUID,
    description={Universally Unique Identifier. Eine UUID ist eine 128-Bit Zahl, die in der Datenbank als Primärschlüssel verwendet wird. Sie ist eindeutig und wird automatisch generiert}
}

\newglossaryentry{varchar}{
    name=varchar,
    description={Datentyp für Zeichenketten. Die maximale Länge einer Zeichenkette kann angegeben werden}
}

\newglossaryentry{bool}{
    name=boolean,
    description={Datentyp für Wahrheitswerte. Kann entweder wahr oder falsch sein}
}

\newglossaryentry{JSON}{
    name=JSON,
    description={JavaScript Object Notation. Ein kompaktes Datenformat, das für Menschen einfach lesbar ist und von Maschinen einfach interpretiert werden kann}
}