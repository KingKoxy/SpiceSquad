\documentclass{implementierungsheft}
% glossary
\makeglossaries

\begin{document}
%Glossary entry
\newglossaryentry{squad}{
    name=Squad,
    description={Cooler Begriff für Gruppe}
}

\newglossaryentry{Admin}{
    name=Admin,
    description={Rolle, die innerhalb einer Gruppe eingenommen werden kann. Admins können Nutzer kicken, bannen, zu Admins befördern sowie Rezepte für Gruppenmitglieder unsichtbar machen. Der Ersteller einer Gruppe ist automatisch ein Admin. Admins können von anderen Admins zu normalen Nutzern degradiert werden.}
}

\newglossaryentry{privat}{
    name=privat,
    description={Ein privates Rezept ist nur für den Nutzer, der es erstellt hat, sichtbar.}
}

\newglossaryentry{Rezept}{
    name=Rezept,
    description={Ein Rezept besteht aus einer Liste an Zutaten mit Mengenangaben sowie einer Zubereitungsanweisung in Fließtext. Zusätzlich können Angaben zum Zeitaufwand und zur Schwierigkeit (Hinzufügedatum?) enthalten sein.}
}
\maketitle
\tableofcontents
\newpage

\section*{Gender-Hinweis}
Zur besseren Lesbarkeit wird in diesem Entwurfsheft das generische Maskulinum verwendet.
Die in diesem Heft verwendeten Personenbezeichnungen beziehen sich - sofern nicht anders kenntlich gemacht - auf alle Geschlechter.
\newpage

\section{Umsetzung der Zielbestimmungen des Pflichtenhefts}
Es wurden alle Muss-, Soll- und Kannkriterien umgesetzt. Die Applikation wurde um die Folgenden Features erweitert:
\begin{itemize}
    \item An sinnvollen Stellen wurden zusätzliche Pop-Up-Fenster eingefügt, die z.B. nach Bestätigung einer Aktion fragen.
    \item Es gibt eine weitere Ansicht, die die exportierten Rezepte als PDF-Vorschau anzeigt.
    \item Die App ist je nach Systemsprache auf Deutsch oder Englisch übersetzt.
\end{itemize}
\newpage
\section{Änderungen am Entwurf}
In den folgenden UML-Diagrammen werden abgeänderte Methoden und Attribute blau markiert, entfernte rot und neue grün.
\subsection{Änderungen am Modellayer}
\begin{figure}[htp]
    \centering
    \includegraphics[width=0.95\textwidth]{images/uml/modellayer.pdf}
    \caption{Änderungen am Modellayer}
    \label{fig:modellayer}
\end{figure}
\paragraph*{\texttt{fromMap(Map<String, dynamic> map)}} Die Factory-Methoden \texttt{fromJson} wurden in \texttt{fromMap} umbenannt, da sie unabhängig von der Serialisierungsmethode sind.
\paragraph*{\texttt{toMap()}} Die statische Methode \texttt{toMap} wurde bei einigen Klassen hinzugefügt, um die Serialisierung zu vereinfachen. Sie gibt ein \texttt{Map<String, dynamic>} zurück, das die Attribute der Klasse enthält.
\paragraph{\texttt{Recipe} und \texttt{RecipeCreationData}}
Die Klasse \texttt{Recipe} wurde in die beiden Klassen \texttt{Recipe} und \texttt{RecipeCreationData} aufgeteilt. Die Klasse \texttt{Recipe} enthält nur noch die Attribute, die ein Rezept nach der Erstellung haben kann, wie zum Beispiel den Author und die Id. Die Klasse erbt von \texttt{RecipeCreationData}, die die Attribute enthält, die ein Rezept bei der Erstellung haben muss, wie Namen oder Anweisungen.
\paragraph{\texttt{Recipe.copyWith(...)}} Die Methode \texttt{copyWith} wurde hinzugefügt, um ein Rezept zu kopieren und dabei einzelne Attribute zu ändern. Die zuändernden Attribute werden als benannte Parameter übergeben.
\paragraph{\texttt{Recipe.createEmpty()}} Die Methode wurde entfernt, da sie nicht mehr benötigt wird.
\paragraph{\texttt{Difficulty.getName(BuildContext context): String}}
Die Methode gibt einen String zurück, der den Namen der Schwierigkeit in der aktuellen Sprache enthält. Die Sprache wird aus dem \texttt{BuildContext} gelesen.
\paragraph{\texttt{Difficulty.fromString(String string)}}
Die statische Methode gibt die Schwierigkeit zurück, die den übergebenen String als Namen hat.
\paragraph{\texttt{RecipeCreationData.imageUrl} und \texttt{Ingredient.iconUrl}} Wir haben uns dazu entschieden statt dem Icon/Bild selbst nur die URL zu speichern. Damit können wir die Anfragengröße massiv verkleinern. Mehr dazu folgt im Kapitel \ref{sec:images}.
\newpage
\subsection{Änderungen am Datalayer}

\section{Glossar}
\printglossary[style=altlist]
\end{document}