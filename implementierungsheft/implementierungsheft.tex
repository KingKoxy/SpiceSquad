\documentclass{implementierungsheft}
% glossary
\makeglossaries

\begin{document}
%Glossary entry
\newglossaryentry{squad}{
    name=Squad,
    description={Cooler Begriff für Gruppe}
}

\newglossaryentry{Admin}{
    name=Admin,
    description={Rolle, die innerhalb einer Gruppe eingenommen werden kann. Admins können Nutzer kicken, bannen, zu Admins befördern sowie Rezepte für Gruppenmitglieder unsichtbar machen. Der Ersteller einer Gruppe ist automatisch ein Admin. Admins können von anderen Admins zu normalen Nutzern degradiert werden.}
}

\newglossaryentry{privat}{
    name=privat,
    description={Ein privates Rezept ist nur für den Nutzer, der es erstellt hat, sichtbar.}
}

\newglossaryentry{Rezept}{
    name=Rezept,
    description={Ein Rezept besteht aus einer Liste an Zutaten mit Mengenangaben sowie einer Zubereitungsanweisung in Fließtext. Zusätzlich können Angaben zum Zeitaufwand und zur Schwierigkeit (Hinzufügedatum?) enthalten sein.}
}
\maketitle
\tableofcontents
\newpage

\section*{Gender-Hinweis}
Zur besseren Lesbarkeit wird in diesem Entwurfsheft das generische Maskulinum verwendet.
Die in diesem Heft verwendeten Personenbezeichnungen beziehen sich - sofern nicht anders kenntlich gemacht - auf alle Geschlechter.
\newpage

\section{Umsetzung der Zielbestimmungen des Pflichtenhefts}
Es wurden alle Muss-, Soll- und Kannkriterien umgesetzt. Die Applikation wurde um die Folgenden Features erweitert:
\begin{itemize}
    \item An sinnvollen Stellen wurden zusätzliche Pop-Up-Fenster eingefügt, die z.B. nach Bestätigung einer Aktion fragen.
    \item Es gibt eine weitere Ansicht, die die exportierten Rezepte als PDF-Vorschau anzeigt.
    \item Die App ist je nach Systemsprache auf Deutsch oder Englisch übersetzt.
\end{itemize}
\newpage
\section{Änderungen am Entwurf}
In den folgenden UML-Diagrammen werden abgeänderte Methoden und Attribute blau markiert, entfernte rot und neue grün.
\subsection{Änderungen am Modellayer}
\begin{figure}[htp]
    \centering
    \includegraphics[width=\textwidth]{images/uml/modelLayer.pdf}
    \caption{Änderungen am Modellayer}
    \label{fig:modellayer}
\end{figure}
\paragraph*{\texttt{fromMap(Map<String, dynamic> map)}} Die Factory-Methoden \texttt{fromJson} wurden in \texttt{fromMap} umbenannt, da sie unabhängig von der Serialisierungsmethode sind.
\paragraph*{\texttt{toMap()}} Die statische Methode \texttt{toMap} wurde bei einigen Klassen hinzugefügt, um die Serialisierung zu vereinfachen. Sie gibt ein \texttt{Map<String, dynamic>} zurück, das die Attribute der Klasse enthält.
\paragraph{\texttt{Recipe} und \texttt{RecipeCreationData}}
Die Klasse \texttt{Recipe} wurde in die beiden Klassen \texttt{Recipe} und \texttt{RecipeCreationData} aufgeteilt. Die Klasse \texttt{Recipe} enthält nur noch die Attribute, die ein Rezept nach der Erstellung haben kann, wie zum Beispiel den Author und die Id. Die Klasse erbt von \texttt{RecipeCreationData}, die die Attribute enthält, die ein Rezept bei der Erstellung haben muss, wie Namen oder Anweisungen.
\paragraph{\texttt{Recipe.copyWith(...)}} Die Methode \texttt{copyWith} wurde hinzugefügt, um ein Rezept zu kopieren und dabei einzelne Attribute zu ändern. Die zuändernden Attribute werden als benannte Parameter übergeben.
\paragraph{\texttt{Recipe.createEmpty()}} Die Methode wurde entfernt, da sie nicht mehr benötigt wird.
\paragraph{\texttt{Difficulty.getName(BuildContext context): String}}
Die Methode gibt einen String zurück, der den Namen der Schwierigkeit in der aktuellen Sprache enthält. Die Sprache wird aus dem \texttt{BuildContext} gelesen.
\paragraph{\texttt{Difficulty.fromString(String string)}}
Die statische Methode gibt die Schwierigkeit zurück, die den übergebenen String als Namen hat.
\paragraph{\texttt{RecipeCreationData.imageUrl} und \texttt{Ingredient.iconUrl}} Wir haben uns dazu entschieden statt dem Icon/Bild selbst nur die URL zu speichern. Damit können wir die Anfragengröße massiv verkleinern. Mehr dazu folgt im Kapitel \ref{sec:images}.
\newpage
\subsection{Änderungen am Datalayer}
\begin{figure}[htp]
    \centering
    \includegraphics[width=\textwidth]{images/uml/dataLayer.pdf}
    \caption{Änderungen am Datalayer}
    \label{fig:dataLayer}
\end{figure}
\subsubsection{\texttt{RecipeRepository}}
Das Konzept, ein lokales und ein entferntes Repository zu haben, wurde verworfen. Stattdessen gibt es nur noch ein Repository, das auf die Daten des Servers zugreift.
\paragraph{\texttt{createRecipe(RecipeCreationData recipe)}} Die Methode nimmt statt einem \texttt{Recipe} ein Objekt der neuen Klasse \texttt{RecipeCreationData} entgegen, das wirklich nur die benötigten Daten enthält.
\paragraph{\texttt{updateRecipe(Recipe recipe)}} Die Methode nimmt keine Id mehr als Parameter entgegen. Stattdessen wird die Id aus dem übergebenen \texttt{Recipe} gelesen.
\paragraph{\texttt{reportRecipe(String recipeId)}} Die Methode nimmt nur noch die Id des zu meldenden Rezepts entgegen und nicht mehr das gesamte Rezept.
\subsubsection{\texttt{AdminRepository}}
\paragraph{\texttt{setCensored(String recipeId, String groupId, bool value)}} Im Entwurf wurde vergessen, die Id der Gruppe als Parameter hinzuzufügen. Das wurde nun nachgeholt.
\subsubsection{\texttt{GroupRepository}}
\paragraph*{\texttt{fetchGroupById()}} Die Methode wurde hinzugefügt, um eine Gruppe anhand ihrer Id zu laden. Dies ist vor allem für die Gurppen-Detail-Seite nötig.
\paragraph*{\texttt{updateGroup(Group group)}} Die Methode nimmt keine Id mehr als Parameter entgegen. Stattdessen wird die Id aus der übergebenen \texttt{Group} gelesen.
\subsubsection{\texttt{UserRepository}}
\paragraph{\texttt{refreshToken}}
Das Attribut wurde entfernt. Stattdessen wird das Token im Systemspeicher abgelegt, da es nur selten benötigt wird, aber über mehrere Sitzungen hinweg gültig ist.
\paragraph{\texttt{getToken(): FutureOr<String?>}} Die Methode gibt ein \texttt{FutureOr<String?>} zurück, da das Token asynchron geladen werden kann, jedoch nicht muss.
\subsubsection{\texttt{IngredientDataRepository}}
Die Klasse wurde umbenannt, da sie nun nicht mehr nur die Namensvorschläge für Zutaten enthält, sondern auch die für Icon-Urls.
\paragraph{\texttt{fetchIngredientIconUrls()}} Die Methode wurde hinzugefügt, um die Liste der möglichen Icons zu laden.
\subsection{Änderungen am Domainlayer}
\section{Glossar}
\printglossary[style=altlist]
\end{document}