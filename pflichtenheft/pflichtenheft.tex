\documentclass[parskip=full]{scrartcl}
\usepackage[top=2.54cm, bottom=2.54cm, left=2.54cm, right=2.54cm]{geometry}
\usepackage[utf8]{inputenc} % use utf8 file encoding for TeX sources
\usepackage[T1]{fontenc}    % avoid garbled Unicode text in pdf
\usepackage[ngerman]{babel}  % german hyphenation, quotes, etc
\usepackage{hyperref}
\usepackage{enumerate}
\usepackage[shortlabels]{enumitem}
\usepackage[dvipsnames]{xcolor}
\usepackage{graphicx}
\usepackage{mwe}
\usepackage{caption}
\usepackage{adjustbox}

\usepackage[headsepline, footsepline]{scrlayer-scrpage}
\addtokomafont{headsepline}{\color{ForestGreen}}
\addtokomafont{footsepline}{\color{ForestGreen}}
\KOMAoptions{headsepline=1.25pt:\textwidth}
\KOMAoptions{footsepline=1.25pt:\textwidth}
\clearpairofpagestyles
\rofoot{\thepage}
\ihead{Title description}

\hypersetup{
    pdftitle={PSE: Pflichtenheft},
    colorlinks,
    linkcolor={black!50!black},
    citecolor={blue!50!black},
    urlcolor={blue!80!black}
}
\usepackage{csquotes}
% detailed hyperlink/pdf configuration
% ‘texdoc hyperref‘ for options
% provides \enquote{} macro for "quotes"

\begin{document}

\begin{titlepage}
    \begin{center}
        \begin{Huge}
            {\textbf{Recishare ()}}
        \end{Huge}
        \vspace{12px}

        Praxis der Softwareentwicklung (PSE)\\
        Sommersemester 2023\\
        \vspace{170px}

        \begin{huge}
            {\textbf{Pflichtenheft}}
        \end{huge}
        \vspace{12px}

        Auftraggeber\\
        Karlsruher Institut für Technologie\\
        KASTEL — Institut für Informationssicherheit und Verlässlichkeit\\
        \vspace{310px}

        Auftragnehmer\\
        Karlsruher Intellektuelle\\
        Henri Becker, Konrad Knappe, Lukas Schwarz, Raphael Zipperer\\
    \end{center}
\end{titlepage}

\tableofcontents

\vspace{32px}
\section*{Gender-Hinweis}
Zur besseren Lesbarkeit wird in diesem Pflichtenheft das generische Maskulinum verwendet.
Die in diesem Heft verwendeten Personenbezeichnungen beziehen sich – sofern nicht anders kenntlich gemacht – auf alle Geschlechter.
\newpage


\section{Zielbestimmung}

\subsection{Musskriterien}
Musskriterien: unabdingbare Leistungen der Software.

\begin{enumerate}[start=1,label={$\langle$\bfseries RM\arabic*$\rangle$}, leftmargin = 5em, itemsep=4pt, parsep=4pt]
    \item Die App besitzt eine Login Funktionen.
    \item Der Nutzer muss Gruppe erstellen/beitreten können.
    \item Der Nutzer muss ein Rezepte erstellen und hochladen können.
    \item Es müssen erstellte Rezepte angeschaut werden können.
    \item Erstellte Rezepte müssen seperat verwalten und geändert werden können.
    \item Der Nutzer muss Gruppen nach dem beitreten wieder verlassen können.
    \item Der Nutzer muss sich wieder ausloggen können
\end{enumerate}

\subsection{Sollkriterien}
Sollkriterien: erstrebenswerte Leistungen.

\begin{enumerate}[start=1,label={$\langle$\bfseries RS\arabic*$\rangle$}, leftmargin = 5em, itemsep=4pt, parsep=4pt]
    \item Der Nutzer soll Rezepte bewerten können
    \item Der Nutzer sol bei Rezepten Portionen skalieren können.
    \item Zu jedem Rezept soll die Kochzeit, Schwierigkeit hinzugefügt werden können
    \item Das Hinzufügedatum eines Rezept soll einsehbar sein.
    \item Man soll zwischen Dark Mode und Light Mode wechseln können.
\end{enumerate}

\subsection{Kannkriterien}
Kannkriterien: Leistungen, die enthalten sein können.

\begin{enumerate}[start=1,label={$\langle$\bfseries RC\arabic*$\rangle$}, leftmargin = 5em, itemsep=4pt, parsep=4pt]
    \item Rezepte können in Form von PDFs exportiert werden.
    \item Es kann ein Bild zum Rezept hochladen werden.
    \item Ein Nutzer kann Rezepte abspeichern oder favoritisieren
    \item Kalorienrechner
    \item Lebensmittelpicker
    \item Einkaufsliste mit Zutaten von Rezepten erstellen
\end{enumerate}

\subsection{Abgrenzungskriterien}
Abgrenzungskriterien: Leistungen die explizit nicht umgesetzt werden.

\begin{enumerate}[start=1,label={$\langle$\bfseries RW\arabic*$\rangle$}, leftmargin = 5em, itemsep=4pt, parsep=4pt]
    \item Es können keine Lebensmittel bestellt werden
    \item Es können keine Kommentare zu Rezepten hinzugefügt werden
    \item Es gibt keine integrierte Chatfunktion
    \item Es können keine Videos den Rezepten hinzugefügt werden
    \item Es gibt keine Moderationsrolle in Gruppen
\end{enumerate}

\section{Produkteinsatz}
Dieses Kapitel dient dazu, den Einsatzbereich, die Zielgruppen und die Betriebsbedingungen der zu entwickelnden Software aufzuführen.

\subsection{Anwendungsbereiche}
In diesem Abschnitt wird erläutert, in welchen Bereichen die Software eingesetzt werden soll.

\subsection{Zielgruppen}
Im Folgenden wird aufgezählt, für welche Anwender die Software im Wesentlichen gedacht ist.

\subsection{Betriebsbedingungen}
In diesem Unterkapitel wird auf die unterschiedlichen Bedürfnisse und Anforderungen an die Software eingegangen.

\section{Produktübersicht}
In diesem Kapitel werden die Produktfunktionen beschrieben und in einem Use-Case-Diagramm visualisiert.
Das Use-Case-Diagramm zeigt mittels Verbindungslininen, wie die einzelnen Use-Cases zueinander stehen.
Der Abschnitt „Gruppenverwaltung“ und "Rezepte Liste" werden genauer in jeweils einem Aktivitätsdiagramm abgebildet.
Dadurch können die einzelnen Schritte, die durchlaufen werden, besser beschrieben werden.

\textbf{Abbildung 3.1}\\
In dem Use-Case-Diagramm „Aufbau der App“ sieht man die allgemeine Struktur der App.
Bei der erstmaligen Nutzung muss der Nutzer sich registrieren oder einlogen, sofern er schon ein Konto.
Dafür muss er seinen Namen, seine E-Mail-Adresse und ein Passwort eingeben.
Nachdem dies geschehen ist, kann der Nutzer auf der Startseite zwischen drei verschiedenen Menüpunkten wählen.

Der Nutzer kann sich die Liste von Rezepten anschauen, dort sind alle Rezepte zu finden, die der Nutzer oder andere Nutzer in beigetretenen Gruppen erstellt haben.
Ein weiterer Menüpunkt sind die Favoriten.
Hier werden alle fovorisierten Rezepte hinterlegt.
Unter dem Menüpunkt "Verwaltung" kann der Nutzer Gruppen erstellen, beitreten oder austreten, sofern er in einer Gruppe ist.
Das bearbeiten von eigenen Rezepten ist hier möglich.
Zudem kann der Nutzer sich hier wieder ausloggen.

\textbf{Abbildung 3.2}\\
Im Aktivitätsdiagramm „Rezepte Liste“ werden die verschiedenen Möglichkeiten unter dem Menüpunkt aufgeführt.
Der Nutzer hat die Möglichkeite Rezepte durch eine Suchfeld zu suchen.
Der Nutzer hat die Möglichkeit die Rezepteliste durch vorgegebene Sortierungen zu sortieren.
Beim Auswählen eines Rezepts öffnet sich das zugehörige Rezeptfenster.
Im Rezeptfenster kann der Nutzer die Portionsgröße anpassen, das Rezept bewerten und das Rezept fovorisieren.
Im Fenster Rezeptliste kann der Nutzer Rezepte erstellen.
Es öffnet sich ein leeres Rezepttemplate Fenster.
Der Nutzer kann die Zutaten und die Kochanleitung in diesem Fenster eingeben.
Nach Bestätigung der Eingabe kommt der Nutzer wieder auf die Rezept Liste.

\textbf{Abbildung 3.3}\\
Im Aktivitätsdiagramm Verwaltung  möchte der Nutzer seine Gruppen verwalten.
Es besteht die Wahl eine Gruppe zu erstellen, eine Gruppe hinzuzufügen oder bestehende Gruppen zu verwalten.
Beim erstellen einer Gruppe kann der Nutzer einen Gruppennamen und ein Passwort wählen.
Nach Bestätigung wird der Nutzer der Gruppe hinzugefügt.
Die Gruppe ist daraufhin unter Verwaltung zu finden.
Beim hinzufügen einer Gruppe muss der Nutzer das Gruppen-Kürzel und Passwort eingeben.
Bei erfolgreicher Eingabe wird der Nutzer der Gruppe hinzugefügt.
Bei fehlerhafter Eingabe erscheint eine Fehlermeldung.
Falls der Nutzer schon in einer Gruppe beigetreten ist, kann er der Gruppe austreten oder das Gruppen-Kürzel und Passwort der Gruppe kopieren.
Beim Austreten muss der Nutzer noch nachträglich bestätigen, dass er aus der austreten möchte.
\newpage

\begin{figure}[!htp]
    \centering
    \begin{adjustbox}{right=160mm}
        \includegraphics[height=220mm]{images/section3/Use-Case-Diagramm Aufbau App.png}
    \end{adjustbox}
    Abbildung 3.1: Use-Case-Diagramm: Aufbau der App
    \label{fig:A31}
\end{figure}
\newpage

\begin{figure}[!htp]
    \centering
    \includegraphics{images/section3/Aktivitaetsdiagramm Gruppenverwaltung.png}\\
    Abbildung 3.2: Aktivitätsdiagramm: Rezepte Liste
    \label{fig:A32}
\end{figure}
\newpage

\begin{figure}[!htp]
    \centering
    \includegraphics{images/section3/Aktivitaetsdiagramm Rezepte Liste.png}\\
    Abbildung 3.3: Aktivitätsdiagramm: Gruppenverwaltung
    \label{fig:A33}
\end{figure}
\newpage

\section{Produktfunktion}
\textbf{Einloggen $\langle$F10$\rangle$}\\
\textbf{Anwendungsfall:} Der Nutzer möchte sich in der App anmelden um alle Features der App nutzen zu können.\\
\textbf{Anforderungen:} RM1\\
\textbf{Ziel:} Der Nutzer erhält Zugang zu seinem Profil in der App indem eine Verbindung zum Server hergestellt wird.\\
\textbf{Vorbedingung:} Der Nutzer hat bereits ein Konto erstellt.\\
\textbf{Nachbedingung Erfolg:} Der Nutzer ist erfolgreich eingeloggt und wird zur Home View der App weitergeleitet.\\
\textbf{Nachbedingung Fehlschlag:} Das Passwort oder der Nutzername ist falsch, sodass ein Pop-up-Fenster erscheint mit einer Fehlermeldung.\\
\textbf{Akteure:} Nutzer, Server\\
\textbf{Auslösendes Ereignis:} Der Nutzer startet die App.\\
\textbf{Beschreibung:}
\begin{enumerate}
    \item Start der App.
    \item Login View erscheint
    \item Eingabe des Nutzernamens.
    \item Eingabe des Passworts.
    \item Auf "Weiter" Button klicken, um Anmeldung abzuschließen.
\end{enumerate}
\textbf{Erweiterung:} -\\
\textbf{Alternativen:} Mit Klick auf den "Registrieren" Button kann ein Nutzer Konto angelegt werden.\\
\newpage

\textbf{Registrierung $\langle$F20$\rangle$}\\
\textbf{Anwendungsfall:} Der Nutzer möchte ein Nutzer Konto anlegen.\\
\textbf{Anforderungen:} - \\
\textbf{Ziel:} Der Nutzer legt ein Nutzer Konto an um die all Features der App nutzen zu können.\\
\textbf{Vorbedingung:} Die App muss bereits gestartet worden sein.
\textbf{Nachbedingung Erfolg:} Der Nutzer wird zum Login Screen weitergeleitet und mit einem Pop-up-Fenster erfolgt die Bestätigung zur erfolgreichen Registrierung.\\
\textbf{Nachbedingung Fehlschlag:} Ein Pop-up-Fenster erscheint mit einer Fehlermeldung.\\
\textbf{Akteure:} Nutzer, Server\\
\textbf{Auslösendes Ereignis:} Der Nutzer muss auf den "Registrieren" Button in der Login View klicken.\\
\textbf{Beschreibung:}
\begin{enumerate}
    \item Klick auf den "Registrieren" Button.
    \item Eingabe des Nutzernamens.
    \item Eingabe eines Passworts.
    \item Wiederholen des Passworts.
    \item Klick auf den "Weiter" Button.
\end{enumerate}
\textbf{Erweiterung:} -\\
\textbf{Alternativen:} Durch das klicken auf den "Login" Button wird der Nutzer zurück zur Login View geleitet.\\
\newpage

\textbf{Rezept erstellen $\langle$F30$\rangle$}\\
\textbf{Anwendungsfall:} Der Nutzer möchte ein neues Rezept erstellen.\\
\textbf{Anforderungen:} RM3, RS3, RC2\\
\textbf{Ziel:} Der Nutzer erstellt ein Rezept, welches er mit seinen Gruppe teilen möchte, oder auch nur für sich sichtbar macht.\\
\textbf{Vorbedingung:} Der Nutzer muss in der App angemeldet sein.\\
\textbf{Nachbedingung Erfolg:} Das neue Rezept kann bei den eigenen Rezepten gefunden werden und kann mit seinen Gruppen geteilt werden.\\
\textbf{Nachbedingung Fehlschlag:} Ein Pop-up-Fenster erscheint mit einer Fehlermeldung.\\
\textbf{Akteure:} Nutzer, Server\\
\textbf{Auslösendes Ereignis:} In der Home View klickt der Nutzer auf den Button für ein neues Rezept anlegen Button.\\
\textbf{Beschreibung:}
\begin{enumerate}
    \item Der Nutzer klickt auf den Button für ein neues Rezept.
    \item Der Name des Rezepts wird eingeben.
    \item Der gesamten Zeitaufwand wird eingeben.
    \item Der Schwierigkeitsgrad wird auswählen.
    \item Die Zutaten mit Mengenangaben werden ausgewählt.
    \item Der Text mit Schritt für Schritt Anleitung wird eingeben.
    \item Die Sichtbarkeit wird auswählen.
    \item Es wird ausgewählt für wie viele Personen das Rezept ist.
    \item Der Nutzer klickt auf den Button und das Rezept wird abgespeichert.
\end{enumerate}
\textbf{Erweiterung:} Der Nutzer kann ein Bild vom fertigen Gericht hochladen.\\
\textbf{Alternativen:} Der Nutzer kann auf den Verlassen Button klicken und das Rezept erstellen wird abgebrochen.\\
\newpage

\textbf{Rezept ändern $\langle$F40$\rangle$}\\
\textbf{Anwendungsfall:} Der Nutzer möchte den Inhalt seines Rezepts nachträglich ändern.\\
\textbf{Anforderungen:} RM5, RS3\\
\textbf{Ziel:} Der Nutzer will sein Rezept anpassen um es an nach seinen Vorstellungen umzugestalten.\\
\textbf{Vorbedingung:} Der Nutzer hat bereits ein eigenes Rezept erstellt.\\
\textbf{Nachbedingung Erfolg:} Das aktualisierte Rezept wird wieder auf den Server geladen und den eingestellten Nutzern aktualisiert angezeigt. \\
\textbf{Nachbedingung Fehlschlag:} Ein Pop-up-Fenster erscheint mit einer Fehlermeldung.\\
\textbf{Akteure:} Nutzer, Server\\
\textbf{Auslösendes Ereignis:} Ein Klick auf den Button zum Rezept ändern in der Recipe View von einem selbst erstellten Rezept.\\
\textbf{Beschreibung:}
\begin{enumerate}
    \item Weiterleitung zum Recipe Creation View, welche bereits mit allen Angaben des Rezepts gefüllt sind.
    \item Änderung des Rezeptnamens, des gesamten Zeitaufwandes, des Schwierigkeitsgrad, der Zutaten zuzüglichen der Mengenangeben, die Anleitung, die Sichtbarkeit oder die Angabe für wie viele Personen das Rezept ist.
    \item Klick auf den Button werden die Änderungen am Rezept gespeichert.
\end{enumerate}
\textbf{Erweiterung:} -\\
\textbf{Alternativen:} Mit einem Klick auf den Verlassen Button kann das Rezept ändern abgebrochen werden, wobei alle Änderungen verloren gehen.\\
\newpage

\textbf{Rezept anschauen $\langle$F50$\rangle$}\\
\textbf{Anwendungsfall:} Der Nutzer möchte sich das Rezepte genauer betrachten.\\
\textbf{Anforderungen:} RM4, RS2, RS4, RC1\\
\textbf{Ziel:} Der Nutzer kann mithilfe der Anleitung und den Mengenangaben das Rezept nachvollziehen.\\
\textbf{Vorbedingung:} Der Nutzer muss Rezepte in der Home View angezeigt bekommen.\\
\textbf{Nachbedingung Erfolg:} Das Rezept wird mit all seinen Details angezeigt in der Recipe View.\\
\textbf{Nachbedingung Fehlschlag:} Ein Pop-up-Fenster erscheint mit einer Fehlermeldung.\\
\textbf{Akteure:} Nutzer, Server\\
\textbf{Auslösendes Ereignis:} Klick auf ein Rezept in der Home View.\\
\textbf{Beschreibung:}
\begin{enumerate}
    \item Durch das Rezept scrollen um die gesamte Anleitung zu sehen, falls diese nicht auf den Screen passt.
\end{enumerate}
\textbf{Erweiterung:} Der Nutzer kann die voreingestellte Portionsgröße auf die gewünschte Anzahl ändern. werden. Des Weiter kann ein Rezept auch in der Form eines PDFs exportiert werden.\\
\textbf{Alternativen:} Mit dem "zurück" Button kann der Nutzer zur Home View zurückkehren.\\
\newpage

\textbf{Rezept favorisieren $\langle$F60$\rangle$}\\
\textbf{Anwendungsfall:} Der Nutzer möchte ein Rezept zu seinen Favoriten hinzugefügen.\\
\textbf{Anforderungen:} RC3
\textbf{Ziel:} Der Nutzer kann ein Rezept zu den Favoriten hinzugefügt werden um einen schnelleren Zugriff auf das Rezept zu ermöglcihen.\\
\textbf{Vorbedingung:} Es müssen dem Nutzer Rezepte in der Home View angezeigt werden.\\
\textbf{Nachbedingung Erfolg:} Der Nutzer kann das Rezept in seiner Favorites View finden.\\
\textbf{Nachbedingung Fehlschlag:} Ein Pop-up-Fenster erscheint mit einer Fehlermeldung.\\
\textbf{Akteure:}Nutzer, Server\\
\textbf{Auslösendes Ereignis:} Klick auf den Favorisieren Button.\\
\textbf{Beschreibung:}
\begin{enumerate}
    \item Klick auf den Favorisieren Button.
\end{enumerate}
\textbf{Erweiterung:} Das Favorisieren kann auch wieder entfernt werden durch das erneute klicken auf den favorisieren Button. Des Weiteren kann die Favoriten List auch nach Autor, Rezeptname und Hinzufüge Datum sortiert werden.
\textbf{Alternativen:} Das Favorisieren kann auch in der Home View durch geführt werden durch das Klicken auf den Stern neben dem Rezept.\\
\newpage

\textbf{Rezept suchen $\langle$F70$\rangle$}\\
\textbf{Anwendungsfall:} Der Nutzer möchte alle Rezepte von allen Gruppen in der Home View durchsuchen.\\
\textbf{Anforderungen:} \\
\textbf{Ziel:} Dem Nutzer werden alle Rezepte mit bestimmten Namen in der Home View angezeigt.\\
\textbf{Vorbedingung:} Vom Nutzer selbst oder in seinen Gruppen wurden bereits Rezepte hochgeladen.\\
\textbf{Nachbedingung Erfolg:} Zum eingegeben Begriff passende Rezepte werden in der Home View angezigt.\\
\textbf{Nachbedingung Fehlschlag:} Keine Rezepte werden in der Home View angezeigt, stattdessen eine Fehlermeldung.\\
\textbf{Akteure:} Nutzer, Server\\
\textbf{Auslösendes Ereignis:} Klick auf die Suchzeile.\\
\textbf{Beschreibung:}
\begin{enumerate}
    \item Klick auf die Suchzeile in der Home View.
    \item Eingabe des Suchbegriffs.
    \item Passende Rezepte werden angezeigt und es kann durch die Ergebnisse gescrollt werden.
\end{enumerate}
\textbf{Erweiterung:} Der Nutzer kann die Ergebnisse auch sortieren nach zuletzt hinzugefügt oder alphabetisch nach dem Rezept Namen oder dem Autornamen.\\
\textbf{Alternativen:} Mit dem Klick auf den Button neben dem eingegebenen Suchwort kann das Suchwort entfernt werden und es wird die normale Home View angezeigt.\\
\newpage

\textbf{Gruppe erstellen $\langle$F80$\rangle$}\\
\textbf{Anwendungsfall:} Der Nutzer möchte eine neue Gruppe erstellen mit der die Rezepte geteilt werden können. \\
\textbf{Anforderungen:} RM2 \\
\textbf{Ziel:} Der Nutzer kann mit der neuen Gruppe seine Rezepte teilen.\\
\textbf{Vorbedingung:} Ein Nutzerprofil ist angemeldet und der Nutzer befindet sich in der Settings View.\\
\textbf{Nachbedingung Erfolg:} Neue Gruppe wird in der Settings View angezeigt.\\
\textbf{Nachbedingung Fehlschlag:} Ein Pop-up-Fenster erscheint mit einer Fehlermeldung.\\
\textbf{Akteure:} Nutzer, Server\\
\textbf{Auslösendes Ereignis:} Ein klick auf den Gruppen erstellen Button.\\
\textbf{Beschreibung:}
\begin{enumerate}
    \item Klick auf den neue Gruppe erstellen Button.
    \item Eingabe des Gruppen Namens.
    \item Eingabe des Gruppen Passworts.
    \item Klick auf den speichern Button wird die Gruppe gespeichert.
\end{enumerate}
\textbf{Erweiterung:} -\\
\textbf{Alternativen:} Mit dem Exit Button kann der Gruppen-erstellungs-Prozess abgebrochen werden, dabei werden keine Daten gespeichert.\\
\newpage

\textbf{Gruppe beitreten $\langle$F90$\rangle$}\\
\textbf{Anwendungsfall:} Der Nutzer möchte einer neuen Gruppe beitreten.\\
\textbf{Anforderungen:} RM2 \\
\textbf{Ziel:} Der Nutzer ist der Gruppe beigetreten und kann alle in der Gruppe geteilten Rezepte ansehen.\\
\textbf{Vorbedingung:} Der Nutzer befindet sich in der Settings View.\\
\textbf{Nachbedingung Erfolg:} Ein Gruppe wird in der Liste der Gruppen angezeigt.\\
\textbf{Nachbedingung Fehlschlag:} Ein Pop-up-Fenster erscheint mit einer Fehlermeldung.\\
\textbf{Akteure:} Nutzer, Server\\
\textbf{Auslösendes Ereignis:} Klick auf den Gruppe beitreten Button.\\
\textbf{Beschreibung:}\\
\begin{enumerate}
    \item Klick auf den Gruppe-beitreten Button.
    \item Eingabe des Kürzels der Gruppe.
    \item Eingabe des Passworts der Gruppe.
    \item Klick auf der Gruppe beitreten Button.
\end{enumerate}
\textbf{Erweiterung:} -\\
\textbf{Alternativen:} Mit dem Exit Button kann der Gruppenbeitritts Prozess abgebrochen werden, alle eingegebene Daten gehen verloren.\\
\newpage

\textbf{Gruppe verlassen $\langle$F100$\rangle$}\\
\textbf{Anwendungsfall:} Der Nutzer möchte seine Gruppenmitgliedschaft beenden.\\
\textbf{Anforderungen:} RM6\\
\textbf{Ziel:} Der Nutzer möchte die Rezepte der Gruppe in der Home View nicht mehr angezeigt bekommen.\\
\textbf{Vorbedingung:} Der Nutzer ist Mitglied in einer Gruppe. \\
\textbf{Nachbedingung Erfolg:} Die Gruppe wird nicht mehr in der Settings View angezeigt.\\
\textbf{Nachbedingung Fehlschlag:} Ein Pop-up-Fenster erscheint mit einer Fehlermeldung.\\
\textbf{Akteure:} Nutzer, Server\\
\textbf{Auslösendes Ereignis:} Der Nutzer befindet sich in der Settings View.\\
\textbf{Beschreibung:}
\begin{enumerate}
    \item Klick auf Gruppe verlassen Button.
    \item Pop-up-Fenster erscheint zur Bestätigung des Verlassens.
    \item Mit Klick auf Ja Button wird Gruppe verlassen.
\end{enumerate}
\textbf{Erweiterung:} -\\
\textbf{Alternativen:} Mit dem Exit Button kann der Gruppen verlassen Prozess abgebrochen werden.\\
\newpage

\textbf{Gruppenkürzel und -passwort ausgeben $\langle$F110$\rangle$}\\
\textbf{Anwendungsfall:}  Der Nutzer möchte das Gruppenkürzel und das Gruppenpasswort erhalten.\\
\textbf{Anforderungen:} \\
\textbf{Ziel:} Der Nutzer kann die Daten der Gruppe mit anderen Nutzern teilen um diese die Möglichkeit zu geben der Gruppe beizutreten..\\
\textbf{Vorbedingung:} Der Nutzer ist in einer Gruppe Mitglied.\\
\textbf{Nachbedingung Erfolg:} Ein Pop-up-Fenster erscheint mit den Daten der Gruppe.\\
\textbf{Nachbedingung Fehlschlag:} Ein Pop-up-Fenster erscheint mit einer Fehlermeldung.\\
\textbf{Akteure:} Nutzer, Server \\
\textbf{Auslösendes Ereignis:} Klick auf den Gruppen Information Button.\\
\textbf{Beschreibung:}
\begin{enumerate}
    \item Klick auf den Gruppen Informations Button.
    \item Ein Pop-up-Fenster erscheint mit dem Gruppenkürzel und dem Gruppenpasswort.
\end{enumerate}
\textbf{Erweiterung:} -\\
\textbf{Alternativen:} Mit dem Exit Button kann der Prozess abgebrochen werden.\\
\newpage

\section{Produktdaten}
Um die App nutzen zu können, ist es erforderlich einige Daten zu speichern. Diese werden auf unterschiedlichen Geräten gespeichert:

\textbf{Smartphone-Daten $\langle$D10$\rangle$}
\begin{itemize}
    \item Lorem
    \item Lorem
\end{itemize}

\textbf{Server-Daten $\langle$D20$\rangle$}
\begin{itemize}
    \item Lorem
    \item Lorem
\end{itemize}

\textbf{Sender-Daten $\langle$D30$\rangle$}
\begin{itemize}
    \item Lorem
    \item Lorem
\end{itemize}

\section{Nichtfunktionale Anforderungen}
Im folgenden Kapitel werden die nichtfunktionalen Anforderungen und Qualitätsmerkmale der App definiert.
Anschließend werden die wichtigsten Qualitätsmerkmale operationalisiert und, falls diese nicht als allgemeine Richtlinie (z.B. Standard, Norm usw.) zu Verfügung gestellt werden,
als konkrete Produktanforderungen konkretisiert.

\subsection{Funktinalität}
\begin{tabular}{| c | c | c | c | c |}
    \hline
    \textbf{Pruduktqualität} & \textbf{sehr gut} & \textbf{gut} & \textbf{normal} & \textbf{nicht relevant} \\ \hline
    Angemessenheit           &                   &              &                 &                         \\ \hline
    Richtigkeit              &                   &              &                 &                         \\ \hline
    Interoperabilität        &                   &              &                 &                         \\ \hline
    Ordnungsmäßigkeit        &                   &              &                 &                         \\ \hline
\end{tabular}

\textbf{Angemessenheit}\\
Da jede Softwarekomponente und das Spiel als Ganzes für die zugeschriebene Funktion geeignet sein muss, ist die Angemessenheit als gut einzustufen.

\textbf{Richtigkeit und Ordnungsmäßigkeit}

\textbf{Interoperabilität}\\
Da die App mit Schnittstellen wie bspw. dem Server oder Android-Betriebssystem fehlerfrei kommunizieren muss, um korrekt zu arbeiten, ist eine sehr gute Interoperabilität wichtig.

\subsection{Sicherheit}
\begin{tabular}{| c | c | c | c | c |}
    \hline
    \textbf{Pruduktqualität} & \textbf{sehr gut} & \textbf{gut} & \textbf{normal} & \textbf{nicht relevant} \\ \hline
    Zuverlässligkeit         &                   &              &                 &                         \\ \hline
    Reife                    &                   &              &                 &                         \\ \hline
    Fehlertoleranz           &                   &              &                 &                         \\ \hline
    Wiederherstellbarkeit    &                   &              &                 &                         \\ \hline
\end{tabular}

\textbf{Zuverlässigkeit und Reife}\\
Durch das Durchführen von Tests während der Implementierung können Fehler früh gefunden werden und der fehlerhafte Code verbessert werden.
Durch die zahlreichen Testfälle können wir eine gute Zuverlässigkeit und Reife der App gewährleisten.

\textbf{Fehlertoleranz}\\
Trotz kontinuierlichen Testens während des Entwicklungsprozesses, kann es zur Laufzeit des Spiels zu Fehlern kommen.
Das Ausmaß ist dabei von Fall zu Fall unterschiedlich.
Folglich wird die Fehlertoleranz der App als normal eingestuft.

\subsection{Wiederherstellbarkeit}


\subsection{Benutzbarkeit}
\begin{tabular}{| c | c | c | c | c |}
    \hline
    \textbf{Pruduktqualität} & \textbf{sehr gut} & \textbf{gut} & \textbf{normal} & \textbf{nicht relevant} \\ \hline
    Verständlichkeit         & X                 &              &                 &                         \\ \hline
    Erlernbarkeit            & X                 &              &                 &                         \\ \hline
    Bedienbarkeit            &                   & X            &                 &                         \\ \hline
    Effizienz                &                   & X            &                 &                         \\ \hline
    Zeitverhalten            &                   & X            &                 &                         \\ \hline
    Verbrauchsverhalten      &                   & X            &                 &                         \\ \hline
\end{tabular}

\textbf{Verständlichkeit, Erlernbarkeit und Bedienbarkeit}\\
Die App soll intuitiv zugänglich sein, um ein hürdenfreie User Experience zu garantieren.

\textbf{Effizienz, Zeitverhalten und Verbraucherverhalten}\\
Die Effizienz der App muss als gut eingestuft werden, um ein möglichst langes Nutzererlebnis trotz des begrenzen Energiespeichers des Endgerätes zu realisieren.
Um dieses Ziel zu erreichen, soll die App ein relativ geringes Verbrauchs- und Zeitverhalten haben.


\subsection{Änderbarkeit}
\begin{tabular}{| c | c | c | c | c |}
    \hline
    \textbf{Pruduktqualität} & \textbf{sehr gut} & \textbf{gut} & \textbf{normal} & \textbf{nicht relevant} \\ \hline
    Analysierbarkeit         &                   &              &                 &                         \\ \hline
    Modifizierbarkeit        &                   &              &                 &                         \\ \hline
    Stabilität               &                   &              &                 &                         \\ \hline
    Prüfbarkeit              &                   &              &                 &                         \\ \hline
    Übertragbarkeit          &                   &              &                 &                         \\ \hline
    Anpassbarkeit            &                   &              &                 &                         \\ \hline
    Installierbarkeit        &                   &              &                 &                         \\ \hline
    Konformität              &                   &              &                 &                         \\ \hline
    Austauschbarkeit         &                   &              &                 &                         \\ \hline
\end{tabular}

\textbf{Analysierbarkeit, Modifizierbarkeit, Anpassbarkeit und Austauschbarkeit}

\textbf{Stabilität}
Inhaltliche und designtechnische Änderungen der App dürfen keine Beeinträchtigung auf die Funktionalität der App haben.
Dadurch muss die Stabilität der App mit gut eingestuft werden.

\textbf{Installierbarkeit}
Um die App spielen zu können, muss die App auf dem Android-Smartphone des Nutzers installiert werden.
Dadurch ist eine gute Installierbarkeit unerlässlich.

\textbf{Konformität}


\subsection{Qualitätsanforderungen}
Die oben als am wichtigsten bezeichneten Qualitätsmerkmale werden im Folgenden operatio- nalisiert, d.h. in konkreten Produktanforderungen konkretisiert oder es wird angegeben, welche Richtlinie (z. B. Standard, Norm) einzuhalten ist.

\begin{enumerate}[start=1,label={$\langle$\bfseries Q\arabic*$\rangle$}, leftmargin = 5em, itemsep=4pt, parsep=4pt]
    \item Lorem
    \item Lorem
    \item Lorem
\end{enumerate}

\section{Benutzeroberfläche/Schnittstellen}
In diesem Kapitel wird die Benutzeroberfläche erläutert. Es wird auf die Darstellung der einzel- nen Funktionen sowie die Navigation zwischen diesen eingegangen.

\textbf{Benutzeroberfläche:}

Bei der App handelt es sich um eine Anwendungssoftware für Android-Endgeräte. Die Benutzer-oberfläche stellt daher einen wichtigen Teil des Produktes dar. Diese muss intuitiv und einfach zu bedienen sein, um eine möglichst hohe Benutzerfreundlichkeit zu gewährleisten. Außerdem sollte sie möglichst ansprechend gestaltet sein, um den Nutzer zu motivieren, die App zu verwenden. Beides wurde im UI-Entwurf berücksichtigt. Durch eine geringe Anzahl an gut strukturierten Ansichten und eine klare Farbgebung wird eine einfache Bedienbarkeit gewährleistet. Wichtige Buttons werden durch eine auffällige Farbe hervorgehoben, um die Navigation zu erleichtern. Es soll außerdem einen Lightmode geben, der entsprechend den Systemeinstellungen automatisch aktiviert wird. Nun sollen die einzelnen Ansichten erläutert werden.

\textbf{Registrieren- und Loginansicht} $\langle$UI10$\rangle$

Beim öffnen der App gelangt der Nutzer auf die Registrierenansicht (\ref{fig:A71}). Hier kann er sich mit einem Nutzernamen und Passwort registrieren. Nach erfolgreicher Registrierung gelangt der Nutzer auf die Gruppenansicht $\langle$UI20$\rangle$. Alternativ kann der Nutzer auch auf die Loginansicht (\ref{fig:A72}) wechseln. Hier kann er sich mit seinen Nutzerdaten einloggen, falls er bereits registriert ist. Nach erfolgreichem Login gelangt der Nutzer zur Hauptseite $\langle$UI30$\rangle$.

\begin{figure}[htp]
    \begin{minipage}
        [t]{0.5\textwidth}
        \centering
        \includegraphics[height=80mm]{images/section7/RegisterView.jpg}
        \label{fig:A71}
        \caption{Registrierenansicht}
    \end{minipage}
    \begin{minipage}
        [t]{0.5\textwidth}
        \centering
        \includegraphics[height=80mm]{images/section7/LoginView.jpg}
        \label{fig:A72}
        \caption{Loginansicht}
    \end{minipage}
\end{figure}

\textbf{Gruppenansicht} $\langle$UI20$\rangle$

Auf der Gruppenansicht (\ref{fig:A73}) kann der Nutzer eine neue Gruppe erstellen, indem er den Namen der neuen Gruppe festlegt und auf "Weiter" drückt. Damit wird die Gruppe in der Datenbank angelegt und der Nutzer tritt dieser bei. Anschließend wird er auf die Hauptansicht $\langle$UI30$\rangle$ weitergeleitet. Alternativ kann der Nutzer auch einer bereits bestehenden Gruppe beitreten. Dazu muss er auf "Gruppe beitreten" drücken, woraufhin sich ein Popup-Fenster öffnet. Hier muss er den Gruppencode eingeben, den jede Gruppe besitzt. Existiert der Gruppencode, so wird der Nutzer der Gruppe hinzugefügt. Auch hier wird der Nutzer auf die Hauptansicht $\langle$UI30$\rangle$ weitergeleitet.

\begin{figure}[!htp]
    \centering
    \includegraphics[height=80mm]{images/section7/GroupView.jpg}
    \label{fig:A73}
    \caption{Gruppenansicht}
\end{figure}
\newpage
\textbf{Hauptansicht} $\langle$UI30$\rangle$

In der Hauptansicht (\ref{fig:A74}) kann der Nutzer alle Rezepte aus den Gruppen, in denen er Mitglied ist sehen. Diese kann er nach verschiedenen Faktoren filtern und sortieren. Durch eine Suchleiste kann er außerdem schnell ein gewünschtes Rezept finden. Jedes Rezept wird hier mit Titel, dem ersten Bild, Kochdauer und Schwierigkeitsgrad abgebildet. Durch einen Klick auf das Herzsymbol wird ein Rezept zu den Favoriten hinzugefügt bzw. wieder entfernt. Klickt ein Nutzer auf ein Rezept, so gelangt er zur Rezeptansicht $\langle$UI40$\rangle$. Durch einen Klick auf das Notizblock-Symbol gelangt der Nutzer auf die Rezepterstellenansicht $\langle$UI60$\rangle$.

\begin{figure}[!htp]
    \centering
    \includegraphics[height=80mm]{images/section7/MainView.jpg}
    \label{fig:A74}
    \caption{Hauptansicht}
\end{figure}

Unten am Bildschirm befindet sich eine Navigationsleiste, die es dem Nutzer ermöglicht, zwischen verschiedenen Ansichten zu wechseln. Durch einen Klick auf das Haus-Symbol gelangt der Nutzer auf die Hauptansicht $\langle$UI30$\rangle$. Durch einen Klick auf das Herz-Symbol gelangt der Nutzer auf die Favoritenansicht $\langle$UI50$\rangle$. Durch einen Klick auf das Personen-Symbol gelangt der Nutzer auf die Verwaltungsansicht $\langle$UI80$\rangle$.

\textbf{Rezeptansicht} $\langle$UI40$\rangle$

In der Rezeptansicht (\ref{fig:A75}) wird ein bestimmtes Rezept angezeigt. Es werden Name, Bilder, Zutaten, Zubereitungsanweisungen, Kochdauer, Schwierigkeitsgrad, Autor und Erstellungsdatum angezeigt. Der Nutzer kann durch anpassen der Portionenzahl automatisch die Zutatenmengen errechnen lassen. Durch einen Klick auf das Herzsymbol wird ein Rezept zu den Favoriten hinzugefügt bzw. wieder entfernt. Ist der Nutzer der Autor des Rezepts, so wird oben ein Stiftsymbol angezeigt. Durch einen Klick auf dieses Symbol gelangt der Nutzer auf die Rezepterstellenansicht $\langle$UI60$\rangle$, die bereits mit dem Rezept befüllt ist. Dort kann er das Rezept bearbeiten. Mit Hilfe des Plus-Symbols über den Bildern kann ein Nutzer außerdem Bilder hochladen, die dann für alle Nutzer im Rezept einsehbar sind. Dies geschieht mit Hilfe des betriebssystemeigenen Dateimanagers. Unten am Bildschirm ist wieder die Navigationsleiste zu finden. Außerdem kann der Nutzer wieder zurück zur vorherigen Ansicht gelangen, indem er auf das Zurück-Symbol oben links klickt.

\begin{figure}[!htp]
    \centering
    \includegraphics[height=80mm]{images/section7/RecipeView.jpg}
    \label{fig:A75}
    \caption{Rezeptansicht}
\end{figure}

\textbf{Favoritenansicht} $\langle$UI50$\rangle$

Die Favoritenansicht (\ref{fig:A76}) ist identisch zur Hauptansicht $\langle$UI30$\rangle$, nur dass hier nur die favorisierten Rezepte angezeigt werden. Außerdem gibt es hier keinen Button zum Rezepte erstellen.
\newpage
\begin{figure}[!htp]
    \centering
    \includegraphics[height=80mm]{images/section7/FavouritesView.jpg}
    \label{fig:A76}
    \caption{Favoritenansicht}
\end{figure}

\textbf{Rezepterstellenansicht} $\langle$UI60$\rangle$

In dieser Ansicht (\ref{fig:A77}) kann ein Nutzer ein neues Rezept erstellen oder bearbeiten. Dazu gibt es verschiedene Eingabefelder, die der Nutzer ausfüllen kann. Diese umfassen den Rezeptnamen, Dauer und Schwierigkeitsgrad sowie die Zubereitungsanweisungen. Außerdem kann er Bilder mit Hilfe des betriebssystemeigenen Dateimanagers hochladen, indem er auf das Bild-Symbol klickt. 
Die Zutaten können mit einem Klick auf das Plus-Symbol hinzugefügt werden. Dazu öffnet sich die Zutatenauswahlansicht, in der der Nutzer die Zutat auswählen kann. Die bereits hinzugefügten Zutaten werden in einer Liste angezeigt und können durch einen Klick auf das Kreuz-Symbol entfernt werden. Ganz unten befindet sich ein Knopf zum Speichern des Rezepts. Durch einen Klick auf das Zurück-Symbol oben links gelangt der Nutzer zurück zur vorherigen Ansicht. Auch hier gibt es wieder die Navigationsleiste.

\begin{figure}[!htp]
    \centering
    \includegraphics[height=80mm]{images/section7/RecipeCreationView.jpg}
    \label{fig:A77}
    \caption{Rezepterstellenansicht}
\end{figure}

\textbf{Zutatenauswahlansicht} $\langle$UI70$\rangle$

In dieser Ansicht (\ref{fig:A78}) kann der Nutzer Zutaten erstellen. Dazu schreibt er den Namen der Zutat in das Eingabefeld und wählt Einheit und Menge aus. Während der Nutzer die Zutat eingibt sollen automatisch Zutaten gesucht werden, die zum bisher geschriebenen Text passen und als Vorschlag unter dem Eingabefeld angezeigt werden. Durch Klick auf so einen Vorschlag wird der Zutatenname auf diesen Vorschlag gesetzt. Der Nutzer hat außerdem die Möglichkeit ein Icon für die Zutat aus einem Dropdownmenü zu wählen. Durch den "Hinzufügen"-Button wird die Zutat erstellt und der Nutzer gelangt zurück zur Rezepterstellenansicht $\langle$UI60$\rangle$.

\begin{figure}[!htp]
    \centering
    \includegraphics[height=80mm]{images/section7/IngredientPickerView.jpg}
    \label{fig:A78}
    \caption{Rezepterstellenansicht}
\end{figure}

\textbf{Verwaltungsansicht} $\langle$UI80$\rangle$

Die Verwaltungsansicht dient der Verwaltung des Nutzers und seiner Gruppen. Der Nutzer kann seinen Anzeigenamen ändern. Zudem kann er Gruppen austreten, in dem er auf das Kreuz-Symbol neben der entsprechenden Gruppe drückt. Außerdem kann er den Gruppencode mit Hilfe des Teilen-Symbols teilen. Durch das Plus-Symbol gelangt der Nutzer zur Gruppenansicht $\langle$UI90$\rangle$, in der er eine neue Gruppe erstellen kann oder einer bestehenden Gruppe beitritt. Außerdem sieht der Nutzer seine erstellten Rezepte, die er hier mit dem entsprechenden Button entfernen oder bearbeiten kann. Durch Klick auf das Rezept wird man zur entsprechenden Rezeptansicht $\langle$UI40$\rangle$. Mit dem Teilen-Button soll ein Rezept als PDF exportiert werden können.
Zudem kann der Nutzer sich ausloggen mit dem Symbol oben rechts. Daraufhin wird er zur Loginansicht $\langle$UI10$\rangle$weitergeleitet. Auch hier gibt es wieder die Navigationsleiste.

\begin{figure}[!htp]
    \centering
    \includegraphics[height=80mm]{images/section7/SettingsView.jpg}
    \label{fig:A79}
    \caption{Verwaltungsansicht}
\end{figure}
\newpage
\section{Technische Produktumgebung}
In diesem Kapitel wird die technische Umgebung des Produktes beschrieben.

\subsection{Software}
\begin{itemize}
    \item Entwicklungsumgebung - Android Studio
    \item Implementierungssprache der App - Dart
    \item Client-Betriebssystem - Android 10 oder eine neuere Androidversion
\end{itemize}

\subsection{Hardware}
\begin{itemize}
    \item Standard Smartphone mit Android Betriebssystem
\end{itemize}

\subsection{Produktschnittstellen}
Die Benutzerschnittstelle wird über ein GUI zur Verfügung gestellt.

\section{Glossar}

\end{document}
