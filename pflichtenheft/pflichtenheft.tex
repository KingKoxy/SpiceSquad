\documentclass[parskip=full]{scrartcl}
\usepackage[top=2.54cm, bottom=2.54cm, left=2.54cm, right=2.54cm]{geometry}
\usepackage[utf8]{inputenc} % use utf8 file encoding for TeX sources
\usepackage[T1]{fontenc}    % avoid garbled Unicode text in pdf
\usepackage[ngerman]{babel}  % german hyphenation, quotes, etc
\usepackage{hyperref}       % hyperlinks in document
\usepackage{glossaries}     % glossary 
\usepackage{enumerate}      % advanced enumeration
\usepackage[shortlabels]{enumitem}
\usepackage[dvipsnames]{xcolor}
\usepackage{graphicx}
\usepackage{caption}        % add captions
\usepackage{adjustbox}      % add adjustmentbox
\usepackage{csquotes}

% Set footer and header bar
\usepackage[headsepline, footsepline]{scrlayer-scrpage}
\addtokomafont{headsepline}{\color{BlueViolet}}
\addtokomafont{footsepline}{\color{BlueViolet}}
\KOMAoptions{headsepline=1.25pt:\textwidth}
\KOMAoptions{footsepline=1.25pt:\textwidth}
\clearpairofpagestyles
\rofoot{\thepage}
\ihead{Write your own Android App: SpiceSquad}

% set the hyperlink style in the document
\hypersetup{
    pdftitle={PSE: Pflichtenheft},
    pdfborderstyle={/S/U/W 1},
    colorlinks,
    linkcolor={black!50!black},
    citecolor={blue!50!black},
    urlcolor={blue!80!black}
}

% sets right quotation for "
\MakeOuterQuote{"}

%change figure numbering
\renewcommand{\thefigure}{\thesection.\arabic{figure}}

%add command to hide subsections from toc
\newcommand{\changelocaltocdepth}[1]{%
  \addtocontents{toc}{\protect\setcounter{tocdepth}{#1}}%
  \setcounter{tocdepth}{#1}%
}

\newcommand{\enablesubsectionnumbering}[1]{
    \renewcommand{\thesubsection}{$\langle$#1\arabic{subsection}0$\rangle$}
    \changelocaltocdepth{1} 
}

\newcommand{\resetsubsectionnumbering}{
    \renewcommand{\thesubsection}{\arabic{section}.\arabic{subsection}}
    \changelocaltocdepth{3} 
}

% glossary
\makeglossaries
\renewcommand{\glossarysection}[2][]{}

\begin{document}
%Glossary entry
\newglossaryentry{squad}{
    name=Squad,
    description={Cooler Begriff für Gruppe}
}

\newglossaryentry{Admin}{
    name=Admin,
    description={Rolle, die innerhalb einer Gruppe eingenommen werden kann. Admins können Nutzer kicken, bannen, zu Admins befördern sowie Rezepte für Gruppenmitglieder unsichtbar machen. Der Ersteller einer Gruppe ist automatisch ein Admin. Admins können von anderen Admins zu normalen Nutzern degradiert werden.}
}

\newglossaryentry{privat}{
    name=privat,
    description={Ein privates Rezept ist nur für den Nutzer, der es erstellt hat, sichtbar.}
}

\newglossaryentry{Rezept}{
    name=Rezept,
    description={Ein Rezept besteht aus einer Liste an Zutaten mit Mengenangaben sowie einer Zubereitungsanweisung in Fließtext. Zusätzlich können Angaben zum Zeitaufwand und zur Schwierigkeit (Hinzufügedatum?) enthalten sein.}
}

\begin{titlepage}
    \begin{center}
        \begin{Huge}
            {\textbf{Write your own Android App: SpiceSquad}}
        \end{Huge}
        \vspace{12px}

        Praxis der Softwareentwicklung (PSE)\\
        Sommersemester 2023\\
        \vspace{150px}

        \begin{Huge}
            {\textbf{Pflichtenheft}}
        \end{Huge}
        \vspace{12px}

        \textbf{Auftraggeber}\\
        Karlsruher Institut für Technologie\\
        KASTEL — Institut für Informationssicherheit und Verlässlichkeit\\
        \vspace{330px}

        \textbf{Auftragnehmer}\\
        Karlsruher Intellektuelle\\
        Henri Becker, Konrad Knappe, Lukas Schwarz, Raphael Zipperer\\
    \end{center}
\end{titlepage}

\tableofcontents
\newpage

\section*{Gender-Hinweis}
Zur besseren Lesbarkeit wird in diesem Pflichtenheft das generische Maskulinum verwendet.
Die in diesem Heft verwendeten Personenbezeichnungen beziehen sich – sofern nicht anders kenntlich gemacht – auf alle Geschlechter.
\newpage

\section{Zielbestimmung}
\subsection{Musskriterien}
Musskriterien: unabdingbare Leistungen der Software.

\begin{enumerate}[start=1,label={$\langle$\bfseries RM\arabic*$\rangle$}, leftmargin = 5em, itemsep=4pt, parsep=4pt]
    \item Der Nutzer muss sich mit einer E-Mail-Adresse, einem Nutzernamen und einem Passwort registrieren können.\label{rm:Registering}
    \item Der registrierte Nutzer muss sich mit seiner E-Mail-Adresse und seinem Passwort anmelden, um auf die Funktionen der App zugreifen zu können. \label{rm:Login}
    \item Der angemeldete Nutzer muss sich abmelden können. \label{rm:Logout}
    \item Der Nutzer muss eine Gruppe erstellen können. \label{rm:GroupCreation}
    \item In jeder Gruppe muss mindestens ein \gls{administrator} sein, standardmäßig ist der Gruppengründer auch \gls{administrator}.\label{rm:GroupAdmin}
    \item Der \glslink{administrator}{Gruppenadministrator} muss die Gruppe auflösen können. \label{rm:GroupDeletion}
    \item Der Nutzer muss einer Gruppe mithilfe eines \glslink{gruppenkurzel}{Gruppenkürzels} beitreten können und diese danach auch wieder verlassen können.\label{rm:GroupJoining}
    \item Der Nutzer muss \glslink{rezept}{Rezepte} erstellen und hochladen können. \label{rm:RecipeCreation}
    \item Der \gls{autor} muss zu jedem seiner \glslink{rezept}{Rezept} den Rezepttitel, die Zubereitungszeit, die Zutaten mit Mengenangabe, den \glslink{schwierigkeit}{Schwierigkeitsgrad}, die Portionsanzahl und eine Zubereitungsanweisung angeben können.\label{rm:RecipeContents}
    \item Der Nutzer muss öffentliche \glslink{rezept}{Rezepte} aller \glslink{autor}{Autoren} in der gleichen Gruppe anschauen können.\label{rm:RecipeViewing}
    \item Der \gls{autor} muss seine \glslink{rezept}{Rezepte} separat ändern und löschen können.\label{rm:RecipeManagement}
    \item Jedes Gruppenmitglied muss das \gls{gruppenkurzel} einsehen können.\label{rm:GroupId}
\end{enumerate}

\subsection{Sollkriterien}
Sollkriterien: Leistungen, die enthalten sein sollten, bei Bedarf jedoch auch weggelassen werden können.

\begin{enumerate}[start=1,label={$\langle$\bfseries RS\arabic*$\rangle$}, leftmargin = 5em, itemsep=4pt, parsep=4pt]
    \item Der Nutzer soll bei \glslink{rezept}{Rezepten} Portionen skalieren können.\label{rs:PortionScaling}
    \item Der Nutzer soll \glslink{rezept}{Rezepte} \glslink{favorit}{favorisieren} können.\label{rs:RecipeFavourites}
    \item Der Nutzer soll seinen \glslink{rezept}{Rezepten} \gls{labels} hinzufügen können.\label{rs:RecipeLabels}
    \item Der \gls{autor} soll die \gls{sichtbarkeit} seiner \glslink{rezept}{Rezepte} ändern können.\label{rs:RecipeVisibility}
    \item Ein \gls{administrator} soll Gruppenmitglieder zum \gls{administrator} ernennen und den \glslink{administrator}{Administratorenstatus} wieder entziehen können.\label{rs:AdminCreation}
    \item Ein \gls{administrator} soll Nutzer aus Gruppen \gls{kicken} und \gls{bannen} können.\label{rs:Kicking}
    \item Ein \gls{administrator} soll \glslink{rezept}{Rezepte} von anderen Nutzern in der Gruppe \gls{ausblenden} und \gls{einblenden} können.\label{rs:RecipeHiding}
    \item Das Hinzufügedatum und der Benutzername des \glslink{autor}{Autors} eines \glslink{rezept}{Rezeptes} soll einsehbar sein.\label{rs:AuthorAndDate}
    \item Der Nutzer soll sein Passwort wieder zurücksetzen können, indem er einen Link an seine E-Mail-Adresse geschickt bekommt.\label{rs:ResetPassword}
    \item Der Nutzer soll sein Konto nach Erstellen wieder löschen können.\label{rs:AccountDeletion}
    \item Der Nutzer soll das \gls{gruppenkurzel} auch über einen QR-Code teilen können, der zum Gruppenbeitritt gescannt werden kann.\label{rs:QRCode}
    \item Der Nutzer soll seinen Nutzernamen ändern können.\label{rs:ChangeUsername}
    \item Der Nutzer soll nach \glslink{rezept}{Rezepten} anhand ihres Titels suchen können.\label{rs:Searching}
    \item Die Gruppe soll aus der Datenbank gelöscht werden, wenn das letzte Mitglied der Gruppe diese verlassen hat.\label{rs:EmptyGroup}
    \item Das älteste verbleibende Mitglied der Gruppe soll \gls{administrator} werden, wenn der einzige \gls{administrator} einer Gruppe diese verlässt.\label{rs:NoAdmins}

\end{enumerate}

\subsection{Kannkriterien}
Kannkriterien: Leistungen, die umgesetzt werden können, jedoch nachrangig sind.

\begin{enumerate}[start=1,label={$\langle$\bfseries RC\arabic*$\rangle$}, leftmargin = 5em, itemsep=4pt, parsep=4pt]
    \item Der Nutzer kann \glslink{rezept}{Rezepte} in Form von PDFs exportieren.\label{rc:PDFExport}
    \item Der \gls{autor} kann genau ein Bild zu jedem seiner \glslink{rezept}{Rezepte} hochladen.\label{rc:Images}
    \item Der Nutzer kann beim Erstellen eines Rezepts Zutaten aus einer Liste suchen und auswählen.\label{rc:IngredientList}
    \item Der Nutzer kann sein Profilbild setzen und ändern.\label{rc:ProfileImage}
    \item Der Nutzer kann \glslink{rezept}{Rezepte} nach \glslink{favorit}{Favoriten}, \glslink{schwierigkeit}{Schwierigkeitsgrad} und \gls{labels} filtern.\label{rc:Filtering}
    \item Der Nutzer kann \glslink{rezept}{Rezepte} jeweils auf- und absteigend nach dem Namen des Rezepts, nach dem \glslink{schwierigkeit}{Schwierigkeitsgrad} und dem Hinzufügedatum sortieren.\label{rc:Sorting}
    \item Nur ein \gls{administrator} kann den Gruppennamen ändern.\label{rc:GroupRenaming}
    \item Der Nutzer kann den \glslink{administrator}{Administratoren} \glslink{rezept}{Rezepte} melden.\label{rc:RecipeMelden}
    \item Der Nutzer kann \glslink{favorit}{favorisierte} \glslink{rezept}{Rezepte} nicht weiterhin anschauen, wenn der \gls{autor} die Gruppe verlässt. Tritt der \gls{autor} wieder ein, sind die \glslink{favorit}{favorisierten} \glslink{rezept}{Rezepte} wieder als solche sichtbar.\label{rc:FavoriteAuthorRejoin}

\end{enumerate}

\subsection{Abgrenzungskriterien}
Abgrenzungskriterien: Leistungen die explizit nicht umgesetzt werden.

\begin{enumerate}[start=1,label={$\langle$\bfseries RW\arabic*$\rangle$}, leftmargin = 5em, itemsep=4pt, parsep=4pt]
    \item Der Nutzer kann keine Kommentare zu \glslink{rezept}{Rezepten} abgeben.
    \item Der Nutzer kann keine \glslink{rezept}{Rezepte} bewerten.
    \item Der Nutzer kann keine Videos zu den \glslink{rezept}{Rezepten} hinzufügen.
    \item Der Nutzer kann keine anderen Nutzer suchen.
    \item Gruppen besitzen kein Gruppenbild.
    \item Es gibt keinen Light-Mode.
    \item \glslink{administrator}{Administratoren} können \glslink{bannen}{Banns} auf Gruppenmitglieder nicht rückgängig machen.
\end{enumerate}

\section{Produkteinsatz}
Dieses Kapitel dient dazu, den Einsatzbereich, die Zielgruppen und die Betriebsbedingungen der zu entwickelnden Software aufzuführen.

\subsection{Anwendungsbereiche}
In diesem Abschnitt wird erläutert, in welchen Bereichen die Software eingesetzt werden soll. \par
Die App bietet eine Reihe von Funktionen, die Nutzern eine Vielzahl an Anwendungsmöglichkeiten rund um das Thema Kochen bieten.
Zum einen können Nutzer \glslink{rezept}{Rezepte} hochladen und diese bei Bedarf im Nachhinein bearbeiten.
Dies ermöglicht das Teilen von \glslink{rezept}{Rezepten} mit anderen Nutzern. Eigene \glslink{rezept}{Rezepte} können außerdem auf \glslink{sichtbarkeit}{privat} gestellt werden, sodass nur der \gls{autor} selbst diese sehen kann.
Die App kann also auch als eine Art digitales Kochbuch benutzt werden, um eigene \glslink{rezept}{Rezepte} zu organisieren und bei Bedarf auch von anderen Geräten aufrufen zu können.
Nutzer können außerdem Kochgruppen, sogenannten \glslink{squad}{Squads}, beitreten, um \glslink{rezept}{Rezepte} mit Nutzern, welche ähnliche Interessen haben, zu teilen und die öffentlichen \glslink{rezept}{Rezepte} anderer Nutzer in diesen Gruppen zu betrachten. Außerdem können Nutzer die \glslink{rezept}{Rezepte} ihrer Gruppen nach bestimmten Kriterien filtern, sortieren und nach Schlagwörtern durchsuchen. So können Nutzer gezielt ihre Kochkenntnisse erweitern.
\glslink{rezept}{Rezepte} können zudem skaliert werden, um die richtige Menge für eine bestimmte Anzahl an Personen effizient zu ermitteln.

\subsection{Zielgruppen}
Im Folgenden wird aufgezählt, für welche Anwender die Software im Wesentlichen gedacht ist. \par
\gls{spicesquad} richtet sich an alle, die gerne kochen und ihre Kocherfahrung verbessern möchten.
Die App ist insbesondere für Gruppen geeignet, in denen gekocht wird (Bsp. Wohngemeinschaften), da \glslink{rezept}{Rezepte} effizient geteilt und skaliert werden können.

\subsection{Betriebsbedingungen}
In diesem Unterkapitel wird auf die unterschiedlichen Bedürfnisse und Anforderungen an die Software eingegangen. \par
Um die volle Funktionsfähigkeit der App garantieren zu können, ist eine Internetverbindung erforderlich. Außerdem können QR-Codes für Gruppen nur mit einer funktionierenden Handykamera gescannt werden, auf die die App zugreifen kann.

\newpage
\section{Produktübersicht}
In diesem Kapitel werden die Produktfunktionen beschrieben und in einem Use-Case-Diagramm "Aufbau der App" visualisiert.
Das Use-Case-Diagramm zeigt mittels Verbindungslininen, wie die einzelnen Use-Cases zueinander stehen.
Die Hauptansicht wird genauer im Aktivitätsdiagramm "Hauptansicht" abgebildet.
Die Gruppen-Detail-Ansicht als \gls{administrator} wird in einem weiteren Aktivitätsdiagramm "Gruppen-Detail-Ansicht als \gls{administrator}" dargestellt und erläutert.
Dadurch können die einzelnen Schritte, die durchlaufen werden, besser beschrieben werden.


\subsection{Aufbau der App (\autoref{fig:UseCaseDiagram}):}

In dem Use-Case-Diagramm „Aufbau der App“ sieht man die allgemeine Struktur der App.

Bei der erstmaligen Nutzung muss der Nutzer sich registrieren. Dazu muss er seinen Benutzernamen, seine E-Mail-Adresse und ein Passwort eingeben.
Das Passwort muss durch eine erneute Eingabe bestätigt werden.
Nach Bestätigung der Eingabe kann der Nutzer eine Gruppe erstellen, einer Gruppe beitreten oder diesen Punkt überspringen und vorerst keiner Gruppe beitreten. Jeder Nutzer kann nur eine begrenzte Anzahl an Gruppen erstellen.
Durch das Erstellen einer Gruppe wird man automatisch zum \gls{administrator} der Gruppe.
Der Nutzer kann einer Gruppe durch Eingabe eines \gls{gruppenkurzel} oder durch Scannen eines QR-Codes beitreten.
Der Nutzer kann eine Gruppe durch Eingabe eines Gruppennamens erstellten.

Falls der Nutzer schon ein Nutzerkonto besitzt, kann er sich einloggen.
Dazu muss er seine E-Mail-Adresse und sein Passwort eingeben. Ist der Nutzer einmal angemeldet, so muss er sich erst wieder nach einer bestimmten Zeit anmelden, selbst wenn die App geschlossen wurde.
Hat der Nutzer sein Passwort vergessen, besteht die Möglichkeit das Passwort zurückzusetzen.

Nach dem Anmelden oder Registrieren gelangt der Nutzer auf die Hauptansicht.
Hier hat er die Möglichkeit, \glslink{rezept}{Rezepte} zu filtern, zu \glslink{favorit}{favorisieren}, nach Schlagwörtern zu durchsuchen oder zu sortieren.
Durch Klicken auf ein Rezept gelangt der Nutzer auf die Rezeptansicht.
Die Rezeptansicht wird unter dem Aktivitätsdiagramm "Hauptansicht" (\autoref{fig:ActivityDiagramMainView}) noch genauer beschrieben.

Durch Klicken auf den Rezept-Erstellen-Button gelangt der Nutzer auf die Rezept-Erstellungs-Ansicht.
Hier kann der Nutzer ein neues Rezept erstellen.
Es können der Rezepttitel, die Zubereitungszeit, die Zutaten mit Mengenangabe, der \glslink{schwierigkeit}{Schwierigkeitsgrad}, die Portionsanzahl und eine
Zubereitungsanweisung für das neue Rezept angegeben werden. Rezepttitel und Zutatenbezeichnung sowie Zubereitungsanweisung sind auf eine sinnvolle Anzahl an Zeichen begrenzt. Außerdem ist nur die Angabe begrenzt vieler verschiedener Zutaten möglich. Der \gls{autor} kann auch ein Bild hochladen, das bei Bedarf automatisch auf eine angebrachte Größe skaliert wird. Im Anschluss kann das Rezept veröffentlicht werden.

Durch Klicken auf den Verwaltungsansicht-Button
gelangt der Nutzer auf die Verwaltungsansicht.
In dieser Ansicht kann der Nutzer sich abmelden, sein Nutzerkonto löschen, seinen Nutzernamen ändern und Gruppen verlassen. Durch das Drücken des Gruppen-Hinzufügen-Buttons kann der Nutzer eine neue Gruppe erstellen oder beitreten. Zudem ist es möglich erstellte \glslink{rezept}{Rezepte} zu löschen, als PDF zu exportieren oder auf \gls{privat} zu stellen. Klickt der Nutzer auf ein Rezept, so kann er dieses in der Rezept-Erstellungs-Ansicht bearbeiten.

Von der Verwaltungsansicht kommt der Nutzer durch Klicken auf eine Gruppe auf die jeweilige Gruppen-Detail-Ansicht. Eingeloggte Nutzer können hier alle Gruppenmitglieder und alle \glslink{rezept}{Rezepte} dieser Gruppenmitglieder einsehen. Außerdem kann die Gruppe verlassen, das Gruppenkürzel kopiert oder der jeweilige QR-Code angezeigt werden. Durch das Klicken auf die \glslink{rezept}{Rezepte} gelangt der Nutzer auf die Rezeptansicht des dazugehörigen Rezepts.

\glslink{administrator}{Gruppenadministratoren} können in der Gruppen-Detail-Ansicht Gruppenmitglieder \gls{kicken}, \gls{bannen}, zum \gls{administrator} ernennen, \glslink{administrator}{Administratoren} zu normalen Nutzern degradieren, \glslink{rezept}{Rezepte} von Gruppenmitgliedern für die Gruppe \gls{ausblenden} oder die Gruppe auflösen. Des Weiteren kann nur ein \gls{administrator} den Gruppennamen ändern.

Im Falle eines Verbindungsverlustes zum Server können bereits geladene \glslink{rezept}{Rezepte} weiterhin angezeigt werden. Wird die App ohne Verbindung zum Server gestartet, wird eine Fehlermeldung angezeigt.

\subsection{Hauptansicht (\autoref{fig:ActivityDiagramMainView}):}

Im Aktivitätsdiagramm „Hauptansicht“
werden die verschiedenen Möglichkeiten unter dem Menüpunkt Hauptansicht aufgeführt.

Der Nutzer hat die Möglichkeit \glslink{rezept}{Rezepte} durch ein Suchfeld zu suchen.
Der Nutzer hat die Möglichkeit die Liste nach verschiedenen vorgegebene Attributen zu sortieren oder zu filtern.
Vorgegebene Sortierungen sind auf- oder absteigend alphabetisch nach Namen des Rezepts, nach dem \glslink{schwierigkeit}{Schwierigkeitsgrad} und nach dem Hinzufügedatum.
Die Liste lässt sich nach \glslink{favorit}{Favoriten}, \gls{labels} oder \glslink{schwierigkeit}{Schwierigkeitsgrad} filtern.
Es besteht zudem die Möglichkeit \glslink{rezept}{Rezepte} schon in dieser Ansicht zu \glslink{favorit}{favorisieren}.\newline
Beim Auswählen eines Rezepts öffnet sich die dazugehörige Rezeptansicht.
In dieser Ansicht kann der Nutzer das Rezept ansehen, \glslink{favorit}{favorisieren} und die Portionsgrößen ändern.
Es besteht zudem die Möglichkeit \glslink{rezept}{Rezepte} zu melden.
Ist der Nutzer der \gls{autor} des Rezepts, hat er zudem die Möglichkeit, das Rezept zu bearbeiten.
Wird ein Rezept gemeldet, so werden die \glslink{administrator}{Administratoren} mit einer E-Mail auf die Meldung hingewiesen.\newline
Durch Klicken auf den Zurück-Button gelangt der Nutzer wieder auf die Hauptansicht.\par


\subsection{Gruppen-Detail-Ansicht als \gls{administrator} (\autoref{fig:ActivityDiagramAdmin}):}

Im Aktivitätsdiagramm "Gruppen-Detail-Ansicht als Administrator" wird gezeigt, welche Möglichkeiten der Nutzer als \gls{administrator} einer Gruppe in der Gruppen-Detail-Ansicht hat.

Der \gls{administrator} muss in der Verwaltungsansicht die Gruppe wählen in der er \glslink{administrator}{Gruppenadministrator} ist.
Der \gls{administrator} landet so in der Gruppen-Detail-Ansicht.
Durch Klicken auf ein Rezept kann der \glslink{administrator}{Gruppenadministrator}, \glslink{einblenden}{eingeblendete} \glslink{rezept}{Rezepte} \gls{ausblenden} und \glslink{ausblenden}{ausgeblendete} \glslink{rezept}{Rezepte} \gls{einblenden}.\newline
Beim Auswählen von Gruppenmitgliedern gibt es die Möglichkeit das Gruppenmitglied zu \gls{kicken} oder zu \gls{bannen}.
Der \gls{administrator} muss die jeweilige Aktion vor dem endgültigen Entfernen nocheinmal bestätigen.
Ein \glslink{kicken}{gekickter} Nutzer kann der Gruppe wieder beitreten wohingegen ein \glslink{bannen}{gebannter} Nutzer der Gruppe unwiderruflich nicht mehr beitreten kann.
Ein \gls{administrator} kann ein Gruppenmitglied zu einem \gls{administrator} ernennen.
Ein \gls{administrator} kann einen anderen \gls{administrator} wieder zu einem normalen Nutzer degradieren.

Der \gls{administrator} einer Gruppe kann durch eine nachfolgende Bestätigung die Gruppe verlassen.
Ist anschließend kein weiteres Mitglied mehr in der Gruppe wird die Gruppe aufgelöst. Ein Nutzer kann der Gruppe anschließend nicht mehr beitreten.
Ein \gls{administrator} kann eine Gruppe durch eine nachfolgende Bestätigung auflösen. Ein Nutzer kann der Gruppe anschließend nicht mehr beitreten und alle Mitglieder der Gruppe werden automatisch \glslink{kicken}{gekickt}.\newline
Der \gls{administrator} kann das \gls{gruppenkurzel} durch "Gruppenkürzel kopieren" in die Zwischenablage kopieren. Der \gls{administrator} kann den QR-Code der Gruppe durch "QR-Code anzeigen" ansehen.

\newpage

\begin{figure}[!htp]
    \centering
    \begin{adjustbox}{right=160mm}
        \includegraphics[height=220mm]{images/produktübersicht/UseCaseDiagram.png}
    \end{adjustbox}
    \caption{Use-Case-Diagramm: Aufbau der App}
    \label{fig:UseCaseDiagram}
\end{figure}
\newpage

\begin{figure}[!htp]
    \centering
    \includegraphics{images/produktübersicht/ActivityDiagramMainView.png}
    \caption{Aktivitätsdiagramm: Hauptansicht}
    \label{fig:ActivityDiagramMainView}
\end{figure}
\newpage

\begin{figure}[!htp]
    \centering
    \begin{adjustbox}{right=150mm}
        \includegraphics[height=230mm]{images/produktübersicht/ActivityDiagramAdmin.png}
    \end{adjustbox}

    \caption{Aktivitätsdiagramm: Gruppen-Detail-Ansicht als Administrator}
    \label{fig:ActivityDiagramAdmin}
\end{figure}
\newpage

\section{Benutzeroberfläche}
\setcounter{figure}{0} 
\enablesubsectionnumbering{UI}
In diesem Kapitel wird die Benutzerberfläche erläutert. Es wird auf die Darstellung der einzelnen Funktionen sowie die Navigation zwischen diesen eingegangen. Alle hier gezeigten Abbildungen dienen lediglich der Veranschaulichung und sind nicht als fertige Entwürfe zu verstehen.

\subsection*{Benutzeroberfläche}
Bei der App handelt es sich um eine Anwendungssoftware für Android-Endgeräte. Die Benutzer-oberfläche stellt daher einen wichtigen Teil des Produktes dar. Diese muss intuitiv und einfach zu bedienen sein, um eine möglichst hohe Benutzerfreundlichkeit zu gewährleisten. Außerdem sollte sie möglichst ansprechend gestaltet sein, um den Nutzer zu motivieren, die App zu verwenden. Beides wurde im UI-Entwurf berücksichtigt. Durch eine geringe Anzahl an gut strukturierten Ansichten und eine klare Farbgebung wird eine einfache Bedienbarkeit gewährleistet. Wichtige Buttons werden durch eine auffällige Farbe hervorgehoben, um die Navigation zu erleichtern. Nun sollen die einzelnen Ansichten erläutert werden.

\subsection{Registrierungsansicht}
\label{ui:RegisterView}
Beim Öffnen der App gelangt der Nutzer auf die Registrierungsansicht (\autoref{fig:RegisterView}). Hier kann sich der Nutzer mit seiner E-Mail-Adresse, einem Nutzernamen und einem Passwort registrieren. Nach erfolgreicher Registrierung gelangt der Nutzer auf die Gruppen-Beitritts-Ansicht \ref{ui:GroupJoiningView}. Alternativ kann der Nutzer auch auf die Anmeldungsansicht \ref{ui:LoginView} wechseln.


\subsection{Anmeldungsansicht}
\label{ui:LoginView}
In der Anmeldungsansicht (\autoref{fig:LoginView}) kann sich ein bereits registrierter Nutzer mit seiner E-Mail-Adresse und seinem Passwort anmelden. Nach erfolgreicher Anmeldung gelangt der Nutzer auf die Hauptansicht \ref{ui:MainView}. Alternativ kann der Nutzer auch auf die Registrierungsansicht \ref{ui:RegisterView} wechseln.
Hat ein Nutzer sein Passwort vergessen, so kann er zur Passwort-Zurücksetzen-Ansicht \ref{ui:PasswordResetView} wechseln.

\subsection{Passwort-Zurücksetzen-Ansicht}
\label{ui:PasswordResetView}
Hier (\autoref{fig:PasswordResetView}) kann der Nutzer sein Password zurücksetzen lassen, indem er seine E-Mail-Adresse eingibt. Durch Betätigen des Weiter-Buttons bekommt er eine E-Mail zugeschickt, in der sich ein Link befindet, mit dem er ein neues Passwort vergeben kann. Außerdem wird er in der App auf die Anmeldungsansicht \ref{ui:LoginView} weitergeleitet.

\subsection{Gruppen-Beitritts-Ansicht}
\label{ui:GroupJoiningView}
In dieser Ansicht (\autoref{fig:GroupJoiningView}) kann der Nutzer einer bereits bestehenden Gruppe beitreten. Dazu kann er entweder ein \gls{gruppenkurzel} eingeben oder mit Hilfe der Handykamera einen QR-Code scannen. In dieser Ansicht gibt es einige Unterschiede, je nachdem ob der Nutzer nach der Registrierungsansicht \ref{ui:RegisterView} oder von der Verwaltungsansicht \ref{ui:SettingsView} auf diese Seite gelangt. Im ersten Fall gibt es noch einen Überspringen-Knopf, mit dem der Nutzer keiner Gruppe beitritt, sondern direkt auf die Hauptansicht \ref{ui:MainView} gelangt. Dorthin wird er auch weitergeleitet, wenn er einer Gruppe beigetreten ist. Im zweiten Fall gibt es einen Zurück-Knopf, mit dem der Nutzer auf die Verwaltungsansicht \ref{ui:SettingsView} zurückgelangt. Das ist auch der Fall, nachdem er einer Gruppe beigetreten ist.
Alternativ kann der Nutzer auch eine Gruppe erstellen, indem er zur Gruppen-Erstellungs-Ansicht \ref{ui:GroupCreationView} wechselt.

\subsection{Gruppen-Erstellungs-Ansicht}
\label{ui:GroupCreationView}
In dieser Ansicht (\autoref{fig:GroupCreationView}) kann der Nutzer eine neue Gruppe erstellen. Dazu muss er einen Gruppennamen eingeben. Durch Betätigen des Weiter-Buttons wird die Gruppe erstellt und der Nutzer wieder je nach vorheriger Ansicht auf die Hauptansicht \ref{ui:MainView} oder die Verwaltungsansicht \ref{ui:SettingsView} weitergeleitet. Auch hier gibt es einen Zurück- oder Überspringen-Button, der so wie in \ref{ui:GroupJoiningView} funktioniert. Jeder Nutzer kann nur begrenzt viele Gruppen erstellen.

\subsection{Hauptansicht}
\label{ui:MainView}
In der Hauptansicht (\autoref{fig:MainView}) kann der Nutzer alle \glslink{rezept}{Rezepte} aus den Gruppen, in denen er Mitglied ist, sehen. Diese kann er nach verschiedenen Faktoren filtern und sortieren. Durch eine Suchleiste kann er außerdem schnell ein gewünschtes Rezept finden. Jedes Rezept wird hier mit Titel, dem Bild, Zubereitungszeit und \glslink{schwierigkeit}{Schwierigkeitsgrad} abgebildet. Durch einen Klick auf das Herzsymbol wird ein Rezept zu den \glslink{favorit}{Favoriten} hinzugefügt bzw. wieder entfernt. Klickt ein Nutzer auf ein Rezept, so gelangt er zur Rezeptansicht \ref{ui:RecipeView}.

Unten am Bildschirm befindet sich eine Navigationsleiste, die es dem Nutzer ermöglicht, zwischen verschiedenen Ansichten zu wechseln. Durch einen Klick auf das Haus-Symbol gelangt der Nutzer auf die Hauptansicht \ref{ui:MainView}. Durch einen Klick auf das Notizblock-Symbol gelangt der Nutzer auf die Rezept-Erstellungs-Ansicht \ref{ui:RecipeCreationView}. Durch einen Klick auf das Personen-Symbol gelangt der Nutzer auf die Verwaltungsansicht \ref{ui:SettingsView}.

\subsection{Rezeptansicht}
\label{ui:RecipeView}

In der Rezeptansicht (\autoref{fig:RecipeView}) wird ein bestimmtes \gls{rezept} angezeigt. Es werden der Name, das Bild, die Zutaten, die Zubereitungsanweisungen, die Zubereitungszeit, der \glslink{schwierigkeit}{Schwierigkeitsgrad}, der \gls{autor} und das Erstellungsdatum angezeigt. Der Nutzer kann durch Anpassen der Portionenzahl automatisch die Zutatenmengen errechnen lassen. Durch einen Klick auf das Herzsymbol wird ein Rezept zu den \glslink{favorit}{Favoriten} hinzugefügt bzw. wieder entfernt. Ist der Nutzer der \gls{autor} des Rezepts, so wird oben ein Stiftsymbol angezeigt. Durch einen Klick auf dieses Symbol gelangt der Nutzer auf die Rezept-Erstellungs-Ansicht \ref{ui:RecipeCreationView}, die bereits mit dem Rezept befüllt ist. Dort kann er das Rezept bearbeiten. Unten am Bildschirm ist wieder die Navigationsleiste, wie in \ref{ui:MainView}, zu finden. Außerdem kann der Nutzer wieder zurück zur vorherigen Ansicht gelangen, indem er auf das Zurück-Symbol oben links klickt. Unter dem Rezept gibt es noch einen Melden-Button, mit dem ein Rezept den \glslink{administrator}{Administratoren} gemeldet werden kann.


\subsection{Rezept-Erstellungs-Ansicht}
\label{ui:RecipeCreationView}
In dieser Ansicht (\autoref{fig:RecipeCreationView}) kann ein Nutzer ein neues Rezept erstellen oder ein Bestehendes bearbeiten. Dazu gibt es verschiedene Eingabefelder, die der Nutzer ausfüllen kann. Diese umfassen den Rezeptnamen, Zubereitungszeit, \gls{labels} und \glslink{schwierigkeit}{Schwierigkeitsgrad} sowie die Zubereitungsanweisungen. Rezepttitel und Zutatenbezeichnung sowie Zubereitungsanweisung sind auf eine sinnvolle Anzahl an Zeichen begrenzt. Außerdem ist nur die Angabe begrenzt vieler verschiedener Zutaten möglich. Zusätzlich muss der Nutzer festlegen, für wie viele Portionen das Rezept ausgelegt ist. Außerdem kann er ein Bild mit Hilfe des betriebssystemeigenen Dateimanagers hochladen, indem er auf das Bild-Symbol klickt. Auch die Dateigröße des Bildes ist beschränkt.
Die Zutaten können mit einem Klick auf das Plus-Symbol hinzugefügt werden. Dazu öffnet sich die Zutaten-Auswahl-Ansicht \ref{ui:IngredientPickerView}, in der der Nutzer die Zutat auswählen kann. Die bereits hinzugefügten Zutaten werden in einer Liste angezeigt und können durch einen Klick auf das Kreuz-Symbol entfernt werden. Ganz unten befindet sich ein Knopf zum Speichern des Rezepts. Mit diesem wird der Nutzer auf die Hauptansicht \ref{ui:MainView} weitergeleitet. Bearbeitet der Nutzer eines seiner \glslink{rezept}{Rezepte} und erstellt kein neues, so gibt es einen Zurück-Button, mit dem er auf die vorherige Ansicht gelangt. Auch hier gibt es wieder die Navigationsleiste aus \ref{ui:MainView}.

\subsection{Zutaten-Auswahl-Ansicht}
\label{ui:IngredientPickerView}
In dieser Ansicht (\autoref{fig:IngredientPickerView}) kann der Nutzer Zutaten erstellen. Dazu schreibt er den Namen der Zutat in das Eingabefeld und wählt Einheit und Menge aus. Während der Nutzer die Zutat eingibt sollen automatisch Zutaten gesucht werden, die zum bisher geschriebenen Text passen und als Vorschlag unter dem Eingabefeld angezeigt werden. Durch Klick auf so einen Vorschlag wird der Zutatenname auf diesen Vorschlag gesetzt. Der Nutzer hat außerdem die Möglichkeit ein Icon für die Zutat aus einem Dropdownmenü zu wählen. Durch den Hinzufügen-Button gelangt der Nutzer zurück zur Rezept-Erstellungs-Ansicht \ref{ui:RecipeCreationView}, wo die Zutat hinzugefügt wurde.
In dieser Ansicht (\autoref{fig:IngredientPickerView}) kann der Nutzer Zutaten erstellen. Dazu schreibt er den Namen der Zutat in das Eingabefeld und wählt Einheit und Menge aus. Während der Nutzer die Zutat eingibt sollen automatisch Zutaten gesucht werden, die zum bisher geschriebenen Text passen und als Vorschlag unter dem Eingabefeld angezeigt werden. Durch Klick auf so einen Vorschlag wird der Zutatenname auf diesen Vorschlag gesetzt. Der Nutzer hat außerdem die Möglichkeit ein Icon für die Zutat aus einem Dropdownmenü zu wählen. Durch den Hinzufügen-Button gelangt der Nutzer zurück zur Rezept-Erstellungs-Ansicht \ref{ui:RecipeCreationView}, wo die Zutat hinzugefügt wurde.


\subsection{Verwaltungsansicht}
\label{ui:SettingsView}
Die Verwaltungsansicht dient der Verwaltung des Nutzers, seiner Gruppen und seiner \glslink{rezept}{Rezepte}. Der Nutzer kann seinen Anzeigenamen und sein Profilbild ändern. Zudem kann er Gruppen verlassen, in dem er auf das Kreuz-Symbol neben der entsprechenden Gruppe drückt. Durch das Plus-Symbol gelangt der Nutzer zur Gruppen-Beitritts-Ansicht \ref{ui:GroupJoiningView} in der er einer Gruppe beitreten oder eine neue Gruppe erstellen kann. Drückt ein Nutzer auf eine der Gruppen, so kommt er zur Gruppen-Detail-Ansicht \ref{ui:GroupDetailView}. Außerdem sieht der Nutzer seine erstellten \glslink{rezept}{Rezepte}, die er hier mit dem entsprechenden Button entfernen, als PDF exportieren, löschen oder auf \gls{privat} stellen kann. Durch Klick auf das Rezept wird man zur entsprechenden Rezeptansicht \ref{ui:RecipeView} geleitet.
Zudem kann der Nutzer sich mit dem Symbol oben rechts ausloggen. Daraufhin wird er zur Anmeldungsansicht \ref{ui:LoginView} weitergeleitet. Er kann außerdem sein Konto auflösen, woraufhin er ebenfalls zur Anmeldungsansicht \ref{ui:LoginView} geleitet wird. Auch hier gibt es wieder die Navigationsleiste aus \ref{ui:MainView}.


\subsection{Gruppen-Detail-Ansicht}
\label{ui:GroupDetailView}
In der Gruppen-Detail-Ansicht (\autoref{fig:GroupDetailView}) kann der Nutzer die Mitglieder und \glslink{rezept}{Rezepte} einer Gruppe ansehen. Durch den QR-Code-Button gelangt er zur QR-Code-Ansicht \ref{ui:QRCodeView}. Durch Klick auf das Verlassen-Symbol kann die Gruppe verlassen werden. Mit dem Plus-Symbol neben der Mitglieder-Sektion wird das \gls{gruppenkurzel} in die Zwischenablage kopiert und kann geteilt werden. Durch Klick auf ein Rezept wird der Nutzer zur entsprechenden Rezeptansicht \ref{ui:RecipeView} weitergeleitet. Als \gls{administrator} hat man zudem die Möglichkeit die Gruppe aufzulösen, woraufhin der Nutzer wieder auf die Verwaltungsansicht \ref{ui:SettingsView} gelangt. Zudem kann er Nutzer aus der Gruppe \gls{kicken}, \gls{bannen} oder zu \glslink{administrator}{Administratoren} ernennen bzw. diesen Titel wieder aberkennen. Außerdem kann er \glslink{rezept}{Rezepte} für alle Gruppenmitglieder außer dem \gls{autor} verstecken bzw. wieder \gls{einblenden}. Weiterhin kann er den Gruppennamen ändern, indem er auf das Stift-Symbol drückt. Auch hier gibt es wieder die Navigationsleiste aus \ref{ui:MainView}.



\subsection{QR-Code-Ansicht}
\label{ui:QRCodeView}
In der QR-Code-Ansicht (\autoref{fig:QRCodeView}) wird der QR-Code einer Gruppe angezeigt. Durch den Zurück-Button kommt der Nutzer zurück zur Gruppen-Detail-Ansicht \ref{ui:GroupDetailView}. Auch hier gibt es wieder die Navigationsleiste aus \ref{ui:MainView}.
\newpage

\subsection*{Abbildungen}
\begin{figure}[htp]
    \begin{minipage}
        [t]{0.49\textwidth}
        \centering
        \includegraphics[height=80mm]{images/benutzeroberfläche/RegisterView.jpg}
        \caption{Registrierungsansicht \ref{ui:RegisterView}}
        \label{fig:RegisterView}
    \end{minipage}
    \begin{minipage}
        [t]{0.49\textwidth}
        \centering
        \includegraphics[height=80mm]{images/benutzeroberfläche/LoginView.jpg}
        \caption{Anmeldungsansicht \ref{ui:LoginView}}
        \label{fig:LoginView}
    \end{minipage}
\end{figure}
\begin{figure}[htp]
    \begin{minipage}
        [t]{0.49\textwidth}
        \centering
        \includegraphics[height=80mm]{images/benutzeroberfläche/PasswordResetView.jpg}
        \caption{Passwort-Zurücksetzen-Ansicht \ref{ui:PasswordResetView}}
        \label{fig:PasswordResetView}
    \end{minipage}
    \begin{minipage}
        [t]{0.49\textwidth}
        \centering
        \includegraphics[height=80mm]{images/benutzeroberfläche/GroupJoiningView.jpg}
        \caption{Gruppen-Beitritts-Ansicht \ref{ui:GroupJoiningView}}
        \label{fig:GroupJoiningView}
    \end{minipage}
\end{figure}
\begin{figure}[htp]
    \begin{minipage}
        [t]{0.49\textwidth}
        \centering
        \includegraphics[height=80mm]{images/benutzeroberfläche/GroupCreationView.jpg}
        \caption{Gruppen-Erstellungs-Ansicht \ref{ui:GroupCreationView}}
        \label{fig:GroupCreationView}
    \end{minipage}
    \begin{minipage}
        [t]{0.49\textwidth}
        \centering
        \includegraphics[height=80mm]{images/benutzeroberfläche/MainView.jpg}
        \caption{Hauptansicht \ref{ui:MainView}}
        \label{fig:MainView}
    \end{minipage}
\end{figure}
\begin{figure}[htp]
    \begin{minipage}
        [t]{0.49\textwidth}
        \centering
        \includegraphics[height=80mm]{images/benutzeroberfläche/RecipeView.jpg}
        \caption{Rezeptansicht \ref{ui:RecipeView}}
        \label{fig:RecipeView}
    \end{minipage}
    \begin{minipage}
        [t]{0.49\textwidth}
        \centering
        \includegraphics[height=80mm]{images/benutzeroberfläche/RecipeCreationView.jpg}
        \caption{Rezept-Erstellungs-Ansicht \ref{ui:RecipeCreationView}}
        \label{fig:RecipeCreationView}
    \end{minipage}
\end{figure}
\begin{figure}[htp]
    \begin{minipage}
        [t]{0.49\textwidth}
        \centering
        \includegraphics[height=80mm]{images/benutzeroberfläche/IngredientPickerView.jpg}
        \caption{Zutaten-Auswahl-Ansicht \ref{ui:IngredientPickerView}}
        \label{fig:IngredientPickerView}
    \end{minipage}
    \begin{minipage}
        [t]{0.49\textwidth}
        \centering
        \includegraphics[height=80mm]{images/benutzeroberfläche/SettingsView.jpg}
        \caption{Verwaltungsansicht \ref{ui:SettingsView}}
        \label{fig:SettingsView}
    \end{minipage}
\end{figure}
\begin{figure}[htp]
    \begin{minipage}
        [t]{0.49\textwidth}
        \centering
        \includegraphics[height=80mm]{images/benutzeroberfläche/GroupDetailView.jpg}
        \caption{Gruppen-Detail-Ansicht \ref{ui:GroupDetailView}}
        \label{fig:GroupDetailView}
    \end{minipage}
    \begin{minipage}
        [t]{0.49\textwidth}
        \centering
        \includegraphics[height=80mm]{images/benutzeroberfläche/QRCodeView.jpg}
        \caption{QR-Code-Ansicht \ref{ui:QRCodeView}}
        \label{fig:QRCodeView}
    \end{minipage}
\end{figure}
\newpage
\renewcommand{\theparagraph}{\arabic{section}.\arabic{subsection}.\arabic{subsubsection}.\arabic{paragraph}}
\setcounter{secnumdepth}{3}

\section{Use-Case Beispiele}
\enablesubsectionnumbering{UC}

\subsection{Einloggen und Ausloggen}
\label{uc:LoginLogout}
\textbf{Anwendungsfall:} Der Nutzer möchte sich in der App anmelden und sich danach wieder abmelden.\\
\textbf{Anforderungen:} \ref{rm:Login}, \ref{rm:Logout}\\
\textbf{Ziel:} Der Nutzer erhält Zugang zu den Funktionen der App, indem er sich anmeldet. Danach wird er wieder abgemeldet.\\
\textbf{Vorbedingung:} Der Nutzer hat bereits ein Konto erstellt.\\
\textbf{Nachbedingung Erfolg:} Der Nutzer ist erfolgreich ausgeloggt und wird zur Anmeldungsansicht der App weitergeleitet.\\
\textbf{Nachbedingung Fehlschlag:} Das Passwort oder der Nutzername ist falsch, sodass eine Fehlermeldung bei dem fehlerhaften Eingabefeld erscheint.\\
\textbf{Akteure:} Nutzer\\
\textbf{Auslösendes Ereignis:} Der Nutzer öffnet die App.\\
\textbf{Beschreibung:}
\begin{enumerate}
    \item Der Nutzer gibt seine E-Mail-Adresse ein.
    \item Der Nutzer gibt sein Passwort ein.
    \item Der Nutzer klickt auf den Weiter-Button, um die Anmeldung abzuschließen.
    \item Der Nutzer wird zur Hauptansicht weitergeleitet.
    \item Der Nutzer klickt auf den Verwaltungsansicht-Button.
    \item Der Nutzer klickt auf den Abmelde-Button.
    \item Der Nutzer wird abgemeldet und zur Anmeldungsansicht weitergeleitet.
\end{enumerate}
\textbf{Alternativen:} Mit einem Klick auf den Passwort-Vergessen-Button kann das Passwort eines bestehenden Kontos geändert werden (siehe \ref{uc:ResetPassword}).
\newpage

\subsection{Registrierung und Konto löschen}
\label{uc:RegisteringDeletingAccount}
\textbf{Anwendungsfall:} Der Nutzer möchte ein Nutzerkonto anlegen und danach sein Konto löschen.\\
\textbf{Anforderungen:} \ref{rm:Registering}, \ref{rs:AccountDeletion} \\
\textbf{Ziel:} Der Nutzer legt ein Nutzerkonto an, um die Funktionen der App nutzen zu können und löscht das Nutzerkonto danach.\\
\textbf{Vorbedingung:} Es ist bisher kein Konto mit der E-Mail-Adresse des Nutzers angelegt worden.
\textbf{Nachbedingung Erfolg:} Der Nutzer wird zum Gruppen-Beitritts-Ansicht weitergeleitet und mit einem Pop-Up-Fenster erfolgt die Bestätigung zur erfolgreichen Registrierung.\\
\textbf{Nachbedingung Fehlschlag:} Es erscheint eine Fehlermeldung bei dem fehlerhaften Eingabefeld.\\
\textbf{Akteure:} Nutzer\\
\textbf{Auslösendes Ereignis:} Der Nutzer muss nach Öffnen der App auf den Registrieren-Button in der Anmeldungsansicht klicken.\\
\textbf{Beschreibung:}
\begin{enumerate}
    \item Der Nutzer gibt einen Nutzernamen ein.
    \item Der Nutzer gibt seine E-Mail-Adresse ein.
    \item Der Nutzer gibt ein Passwort ein.
    \item Der Nutzer wiederholt das Passwort.
    \item Der Nutzer klickt auf den Weiter-Button.
    \item Der Nutzer wird in die Gruppen-Beitritts-Ansicht weitergeleitet.
    \item Der Nutzer klickt auf den Überspringen-Button.
    \item Der Nutzer wird zur Hauptansicht weitergeleitet.
    \item Der Nutzer klickt auf den Verwaltungsansicht-Button.
    \item Der Nutzer klickt auf den Konto-Löschen-Button.
    \item Ein Pop-Up-Fenster erscheint zur Bestätigung des Vorgangs.
    \item Der Nutzer klickt auf den Bestätigen-Button.
    \item Das Benutzerkonto wird gelöscht.
    \item Der Nutzer wird zur Anmeldungsansicht weitergeleitet.
\end{enumerate}
\textbf{Alternativen:} Der Prozess des Konto-Löschens kann abgebrochen werden indem eine Bestätigung verweigert wird.
\newpage

\subsection{Passwort zurücksetzen}
\label{uc:ResetPassword}
\textbf{Anwendungsfall:} Der Nutzer möchte das Passwort für seinen Benutzerkonto zurücksetzen.\\
\textbf{Anforderungen:} \ref{rs:ResetPassword}\\
\textbf{Ziel:} Der Nutzer kann ein neues Passwort setzen und sich anschließend damit anmelden.\\
\textbf{Vorbedingung:} Es gibt ein Konto mit der angegebenen E-Mail-Adresse.\\
\textbf{Nachbedingung Erfolg:} Der Nutzer hat sich mit einem neuen Passwort angemeldet.\\
\textbf{Nachbedingung Fehlschlag:} Ein Pop-Up-Fenster erscheint mit einer Fehlermeldung. Das Passwort des jeweiligen Accounts wurde nicht verändert.\\
\textbf{Akteure:} Nutzer\\
\textbf{Auslösendes Ereignis:} Der Nutzer muss nach Öffnen der App auf den Passwort-Vergessen-Button in der Anmeldungsansicht klicken.\\
\textbf{Beschreibung:}
\begin{enumerate}
    \item Der Nutzer gibt seine E-Mail-Adresse ein.
    \item Die Anmeldungsansicht öffnet sich und ein Pop-Up-Fenster mit der Information, dass eine E-Mail versendet wurde, wird angezeigt.
    \item Eine E-Mail mit einem Link zu einer Website auf der ein neues Passwort vergeben werden kann wird an die angegebene E-Mail-Adresse geschickt.
    \item Der Nutzer folgt dem Link.
    \item Der Nutzer gibt ein neues Passwort ein.
    \item Der Nutzer kehrt in die App zurück und meldet sich wie in \ref{uc:LoginLogout} beschrieben an.
\end{enumerate}
\textbf{Alternativen:} Durch das Klicken auf den Zurück-Button in der Passwort-Zurücksetzen-Ansicht wird der Nutzer zurück zur Anmeldungsansicht geleitet.
\newpage


\subsection{Benutzernamen und Profilbild ändern}
\textbf{Anwendungsfall:} Der Nutzer möchte seinen Nutzernamen und sein Profilbild ändern.\\
\textbf{Anforderungen:} \ref{rs:ChangeUsername}, \ref{rc:ProfileImage}\\
\textbf{Ziel:} Der Nutzer kann seinen Benutzernamen und sein Profilbild ändern.\\
\textbf{Vorbedingung:} Der Nutzer ist bereits angemeldet und befindet sich in der Verwaltungsansicht.\\
\textbf{Nachbedingung Erfolg:} Der aktualisierte Nutzername und das aktualisierte Profilbild wird anderen Nutzern angezeigt.\\
\textbf{Nachbedingung Fehlschlag:} Der Nutzername oder das Profilbild wurden nicht ersetzt.\\
\textbf{Akteure:} Nutzer\\
\textbf{Auslösendes Ereignis:} -\\
\textbf{Beschreibung:}
\begin{enumerate}
    \item Der Nutzer klickt auf den Nutzernamen-Bearbeiten-Button.
    \item Der Nutzer gibt den neuen Nutzernamen in einem Pop-Up-Fenster ein.
    \item Der Nutzer klickt auf den Speichern-Button.
    \item Der Nutzer klickt auf sein Profilbild.
    \item Der Nutzer kann mit Hilfe des betriebssystemeigenen Dateimanagers ein neues Profilbild auswählen.
\end{enumerate}
\textbf{Alternativen:} Der Nutzer kann den Prozess des Benutzernamens-Ändern abbrechen, indem er das Pop-Up-Fenster schließt.
\newpage

\subsection{Rezept erstellen, ändern und löschen}
\textbf{Anwendungsfall:} Der Nutzer möchte ein neues Rezept erstellen, es danach ändern sowie ein Bild hochladen und es schließlich löschen.\\
\textbf{Anforderungen:} \ref{rm:RecipeCreation}, \ref{rm:RecipeContents}, \ref{rm:RecipeManagement}, \ref{rc:Images}, \ref{rc:IngredientList}\\
\textbf{Ziel:} Der Nutzer erstellt ein Rezept, welches er mit seinen Gruppen teilt, nimmt dann Änderungen vor und löscht es schließlich.\\
\textbf{Vorbedingung:} Der Nutzer ist angemeldet und befindet sich in der Hauptansicht.\\
\textbf{Nachbedingung Erfolg:} Das erstellte und geänderte Rezept wird in der Verwaltungsansicht nicht mehr angezeigt.  \\
\textbf{Nachbedingung Fehlschlag:} -\\
\textbf{Akteure:} Nutzer\\
\textbf{Auslösendes Ereignis:} In der Hauptansicht klickt der Nutzer auf den Neues-Rezept-Anlegen-Button.\\
\textbf{Beschreibung:}
\begin{enumerate}
    \item Der Nutzer gibt den Rezepttitel, die Zubereitungszeit, die Portionenzahl, den \glslink{schwierigkeit}{Schwierigkeitsgrad} und die Zubereitungsanweisungen ein und wählt verschiedene \gls{labels} aus.
    \item Er drückt auf den Zutaten-Hinzufügen-Button.
    \item Er wird zur Zutaten-Hinzufügen-Ansicht weitergeleitet.
    \item Er tippt den anfang des Namens der Zutat ein und wählt das erste angezeigte Ergebnis aus.
    \item Er gibt die Menge und die Einheit der Zutat ein.
    \item Er drückt auf den Hinzufügen-Button.
    \item Der Nutzer klickt auf den Speichern-Button.
    \item Der Nutzer wird zur Hauptansicht weitergeleitet und das Rezept ist in der Liste sichtbar.
    \item Der Nutzer klickt auf den Verwaltungsansicht-Button.
    \item Der Nutzer klickt auf das gerade erstellte Rezept.
    \item Der Nutzer wird zur Rezept-Erstellungs-Ansicht weitegeleitet.
    \item Der Nutzer ändert den Rezepttitel und lädt ein Bild des fertigen Gerichts hoch.
    \item Der Nutzer klickt auf den Speichern-Button.
    \item Der Nutzer wird zur Hauptansicht weitergeleitet, wo das aktualisierte Rezept in der Liste angezeigt wird.
    \item Der Nutzer klickt auf den Verwaltungsansicht-Button.
    \item Der Nutzer klickt auf den Löschen-Button des gerade geänderten Rezepts.
    \item Ein Pop-Up-Fenster erscheint mit der Frage zur Bestätigung der Aktion.
    \item Der Nutzer klickt auf den Bestätigen-Button.
    \item Keinem Nutzer wird das Rezept mehr angezeigt.
\end{enumerate}
\textbf{Alternativen:} Jeder der zuvor genannten Prozesse kann durch ein Klick auf den Zurück-Button abgebrochen werden.
\newpage

\subsection{Rezept suchen, sortieren, filtern, betrachten und favorisieren}
\textbf{Anwendungsfall:} Der Nutzer möchte nach einem Rezept suchen, dann die \glslink{rezept}{Rezepte} nach dem \glslink{schwierigkeit}{Schwierigkeitsgrad} filtern, sich ein Rezept genauer anschauen und dieses dann \glslink{favorit}{favorisieren}.\\
\textbf{Anforderungen:} \ref{rm:RecipeViewing}, \ref{rs:RecipeFavourites}, \ref{rs:PortionScaling}, \ref{rs:Searching}, \ref{rc:Filtering}, \ref{rc:Sorting}, \ref{rs:AuthorAndDate}, \ref{rs:RecipeLabels}\\
\textbf{Ziel:} Der Nutzer kann mit dem Suchen, dem Filtern und dem \glslink{favorit}{Favorisieren} sein gewünschtes Rezept schnell finden und es mithilfe der Zubereitungsanweisungen nachvollziehen.\\
\textbf{Vorbedingung:} Dem Nutzer wird mindestens ein Rezept in der Hauptansicht oder in der Gruppen-Detail-Ansicht angezeigt, das die entsprechenden Filterkriterien erfüll, noch nicht favorisiert wurde und dem Suchtext entspricht. Der Nutzer ist angemeldet und befindet sich in der Hauptansicht.\\
\textbf{Nachbedingung Erfolg:} Das \glslink{favorit}{favorisierte} Rezept ist mit den anderen \glslink{favorit}{favorisierten} \glslink{rezept}{Rezepten} beim Filter \glslink{favorit}{favorisierte} \glslink{rezept}{Rezepte} auffindbar.\\
\textbf{Nachbedingung Fehlschlag:} -\\
\textbf{Akteure:} Nutzer\\
\textbf{Auslösendes Ereignis:} -\\
\textbf{Beschreibung:}
\begin{enumerate}
    \item Der Nutzer klickt auf die Suchzeile.
    \item Der Nutzer gibt einen Text ein und drückt Enter.
    \item Der Nutzer klickt auf den Sortieren-Button und wählt Hinzufügedatum aus.
    \item Der Nutzer klickt auf den Filtern-Button und wählt \glslink{schwierigkeit}{Schwierigkeitsgrad} Mittel aus.
    \item Der Nutzer klickt auf das erste Rezept.
    \item Der Nutzer wird zur Rezeptansicht weitergeleitet.
    \item Der Nutzer ändert die Portionszahl und die Zutatenmengen werden automatisch entsprechend angepasst.
    \item Der Nutzer klickt auf den Favoriten-Button.
\end{enumerate}
\textbf{Alternativen:} -
\newpage

\subsection{Rezept auf privat stellen und exportieren}
\textbf{Anwendungsfall:} Der Nutzer möchte ein eigenes \gls{rezept} auf \gls{privat} stellen und dann exportieren.\\
\textbf{Anforderungen:} \ref{rm:RecipeViewing}, \ref{rs:RecipeHiding}, \ref{rc:PDFExport} \\
\textbf{Ziel:} Der Nutzer will, dass das Rezept nur ihm sichtbar ist und eine PDF-Datei des Rezepts erzeugen.\\
\textbf{Vorbedingung:} Der Nutzer ist angemeldet und befindet sich in der Verwaltungsansicht. Er hat mindestens ein Rezept erstellt.\\
\textbf{Nachbedingung Erfolg:} Anderen Nutzern ist das Rezept nicht mehr sichtbar und der Nutzer hat die Möglichkeit eine PDF des Rezepts herunterzuladen.\\
\textbf{Nachbedingung Fehlschlag:} -\\
\textbf{Akteure:} Nutzer\\
\textbf{Auslösendes Ereignis:} -\\
\textbf{Beschreibung:}
\begin{enumerate}
    \item Der Nutzer klickt auf das Augen-Symbol neben dem Rezept.
    \item Das Rezept ist nun \gls{privat} und wird anderen Nutzern nicht mehr in der Hauptansicht angezeigt.
    \item Der Nutzer klickt auf den Exportieren-Button.
    \item Eine PDF-Datei des Rezepts wird heruntergeladen.
\end{enumerate}
\textbf{Alternativen:} -
\newpage

\subsection{Gruppe erstellen, Gruppe mit Kürzel beitreten und Gruppe auflösen}
\textbf{Anwendungsfall:} Der Nutzer möchte eine neue Gruppe erstellen, und ein weiterer Nutzer dieser beitreten. Danach löst der Nutzer die Gruppe wieder auf\\
\textbf{Anforderungen:} \ref{rm:GroupCreation}, \ref{rm:GroupAdmin}, \ref{rm:GroupDeletion} \\
\textbf{Ziel:} Der Nutzer kann mit der neuen Gruppe seine \glslink{rezept}{Rezepte} teilen und danach die Gruppe auch wieder auflösen.\\
\textbf{Vorbedingung:} Beide Nutzer sind angemeldet und befinden sich in der Verwaltungsansicht.\\
\textbf{Nachbedingung Erfolg:} Die erstellte Gruppe wird keinem Nutzer mehr in der Verwaltungsansicht angezeigt.\\
\textbf{Nachbedingung Fehlschlag:} -\\
\textbf{Akteure:} Nutzer 1, Nutzer 2\\
\textbf{Auslösendes Ereignis:} Nutzer 1 klickt auf den Gruppe-Hinzufügen-Button.\\
\textbf{Beschreibung:}
\begin{enumerate}
    \item Nutzer 1 wird zur Gruppen-Beitritts-Ansicht weitergeleitet.
    \item Nutzer 1 navigiert zur Gruppen-Erstellungs-Ansicht.
    \item Nutzer 1 gibt den Gruppennamen ein.
    \item Nutzer 1 klickt auf den Speichern-Button. Die Gruppe wird angelegt und Nutzer 1 ist Administrator der Gruppe.
    \item Nutzer 1 wird zur Verwaltungsansicht zurückgeleitet.
    \item Nutzer 1 klickt auf die Gruppe und wird zur Gruppen-Detail-Ansicht weitergeleitet.
    \item Nutzer 1 drückt auf das Teilen-Symbol.
    \item Nutzer 1 schickt das \gls{gruppenkurzel} an Nutzer 2.
    \item Nutzer 2 drückt auf den Gruppe-Hinzufügen-Button.
    \item Nutzer 2 wird zur Gruppen-Beitritts-Ansicht weitergeleitet.
    \item Nutzer 2 gibt das \gls{gruppenkurzel} ein.
    \item Nutzer 2 klickt auf den Weiter-Button.
    \item Nutzer 2 wird zur Verwaltungsansicht weitergeleitet.
    \item Nutzer 1 drückt auf den Gruppe-Auflösen-Button und wird zur Verwaltungsansicht geleitet.
    \item Bei beiden Nutzern verschwindet die Gruppe aus der Verwaltungsansicht.
\end{enumerate}
\textbf{Alternativen:} -
\newpage

\subsection{Gruppe mit QR-Code beitreten und verlassen}
\textbf{Anwendungsfall:} Der Nutzer möchte einer neuen Gruppe mit einem QR-Code beitreten und diese danach verlassen.\\
\textbf{Anforderungen:} \ref{rm:GroupJoining}, \ref{rs:QRCode} \\
\textbf{Ziel:} Nutzer 2 ist der Gruppe beigetreten und hat diese danach wieder verlassen.\\
\textbf{Vorbedingung:} Beide Nutzer sind angemeldet. Nutzer 1 ist Mitglied in einer Gruppe, in der Nutzer 2 nicht ist, und befindet sich in der dazugehörigen Gruppen-Detail-Ansicht. Nutzer 2 befindet sich in der Verwaltungsansicht.\\
\textbf{Nachbedingung Erfolg:} Die beigetretene Gruppe wird nicht mehr in der Verwaltungsansicht angezeigt und die \glslink{rezept}{Rezepte} der Gruppe werden nicht mehr in der Hauptansicht angezeigt.\\
\textbf{Nachbedingung Fehlschlag:} -\\
\textbf{Akteure:} Nutzer 1, Nutzer 2\\
\textbf{Auslösendes Ereignis:} -\\
\textbf{Beschreibung:}
\begin{enumerate}
    \item Nutzer 1 klickt auf den QR-Code-Button.
    \item Nutzer 1 wird zur QR-Code-Ansicht weitergeleitet.
    \item Nutzer 2 klickt auf den Gruppe-Beitreten-Button in der Verwaltungsansicht.
    \item Nutzer 2 wird zur Gruppen-Beitritts-Ansicht weitergeleitet.
    \item Nutzer 2 klickt auf den QR-Code-Scannen-Button.
    \item Die Kamera des Smartphones von Nutzer 2 öffnet sich.
    \item Nutzer 2 scannt den QR-Code von Nutzer 1.
    \item Nutzer 2 wird zur Verwaltungsansicht weitergeleitet und ist jetzt Mitglied in der Gruppe.
    \item Nutzer 2 klickt auf den Gruppe-Verlassen-Button.
    \item Ein Pop-Up-Fenster erscheint, welches um Bestätigung bittet.
    \item Nutzer 2 klickt auf den Button zum Bestätigen.
    \item Die Gruppe verschwindet aus der Gruppenliste von Nutzer 2.
\end{enumerate}
\textbf{Alternativen:} -
\newpage

\subsection{Administrator ernennen und Administratorenstatus entfernen}
\textbf{Anwendungsfall:} Der Nutzer möchte ein weiteres Mitglied der Gruppe zu einem \gls{administrator} der Gruppe ernennen und dem Nutzer den \glslink{administrator}{Administratorenstatus} dann wieder aberkennen.\\
\textbf{Anforderungen:} \ref{rs:AdminCreation}\\
\textbf{Ziel:} Der \gls{administrator} kann seine Rechte für die Gruppe mit anderen Mitgliedern teilen und auch anderen Mitglieder die Rechte entziehen.\\
\textbf{Vorbedingung:} Der Nutzer ist angemeldet und\gls{administrator} der Gruppe. Er befindet sich in der Gruppen-Detail-Ansicht.\\
\textbf{Nachbedingung Erfolg:} Ein Mitglied der Gruppe besitzt die gegebenen Rechte nicht mehr.\\
\textbf{Nachbedingung Fehlschlag:} -\\
\textbf{Akteure:} \gls{administrator}, Nutzer \\
\textbf{Auslösendes Ereignis:} -\\
\textbf{Beschreibung:}
\begin{enumerate}
    \item Der \gls{administrator} klickt den Optionen-Button neben einem Mitglied der Gruppe.
    \item Ein Drop-Down-Menü öffnet sich, aus dem der \gls{administrator} die Option auswählt, den Nutzer zum \gls{administrator} der Gruppe zu ernennen.
    \item Das Drop-Down-Menü schließt sich und der Nutzer ist nun ebenfalls \gls{administrator} der Gruppe und wird als solcher angezeigt.
    \item Der \gls{administrator} klickt den Optionen-Button neben dem neuen Administrator.
    \item Der \gls{administrator} klickt auf die Option den Nutzer den \glslink{administrator}{Administratorenstatus} in der Gruppe zu entziehen.
    \item Das Drop-Down-Menü schließt sich und der Nutzer ist nun kein \gls{administrator} der Gruppe mehr und wird auch nicht als solcher angezeigt.
\end{enumerate}
\textbf{Alternativen:} -
\newpage

\subsection{Nutzer kicken und Nutzer bannen}
\textbf{Anwendungsfall:} Ein \gls{administrator} möchte ein Mitglied der Gruppe aus der Gruppe \gls{kicken} und ein anderes Mitglied aus der Gruppe \gls{bannen}.\\
\textbf{Anforderungen:} \ref{rs:Kicking}\\
\textbf{Ziel:} Ein \gls{administrator} möchte das erste Mitglied nur aus der Gruppe \gls{kicken} und das zweite Mitglied \gls{bannen}.\\
\textbf{Vorbedingung:} Der Nutzer ist \gls{administrator} der Gruppe, angemeldet und befindet sich in der Gruppen-Detail-Ansicht. Alle drei Akteure sind in der gleichen Gruppe.\\
\textbf{Nachbedingung Erfolg:} Die beiden Nutzer sind nicht mehr in der Gruppe.\\
\textbf{Nachbedingung Fehlschlag:} -\\
\textbf{Akteure:} \gls{administrator}, Nutzer 1, Nutzer 2\\
\textbf{Auslösendes Ereignis:} -\\
\textbf{Beschreibung:}
\begin{enumerate}
    \item Der \gls{administrator} klickt auf den Optionen-Button von Nutzer 1.
    \item Der \gls{administrator} klickt auf die Option den Nutzer zu \gls{kicken}.
    \item Ein Pop-Up-Fenster erscheint mit einer Bitte um Bestätigung, dass der Nutzer \glslink{kicken}{gekickt} werden soll.
    \item Der \gls{administrator} klickt auf den Bestätigen-Button.
    \item Nutzer 1 verschwindet aus der Liste der Gruppenmitglieder und ist kein Mitglied der Gruppe mehr.
    \item Der \gls{administrator} klickt auf den Optionen-Button von Nutzer 2.
    \item Der \gls{administrator} klickt auf die Option den Nutzer zu \gls{bannen}.
    \item Ein Pop-Up-Fenster erscheint mit einer Bitte um Bestätigung, dass der Nutzer \glslink{bannen}{gebannt} werden soll.
    \item Der \gls{administrator} klickt auf den Bestätigen-Button.
    \item Nutzer 2 verschwindet aus der Liste der Gruppenmitglieder und ist kein Mitglied der Gruppe mehr. Er kann ihr auch nicht mehr beitreten.
\end{enumerate}
\textbf{Alternativen:} -
\newpage


\subsection{Rezepte verwalten für die Gruppen}
\textbf{Anwendungsfall:} Der Nutzer meldet ein Rezept, welches mit der Gruppe von einem Nutzer geteilt wird, und der \gls{administrator} der Gruppe will dieses Rezept \gls{ausblenden}\\
\textbf{Anforderungen:} \ref{rs:RecipeHiding}, \ref{rc:RecipeMelden}\\
\textbf{Ziel:} Der \gls{administrator} kann bestimmen welche \glslink{rezept}{Rezepte} den Mitgliedern der Gruppe und ihm selbst angezeigt werden.\\
\textbf{Vorbedingung:} Der Nutzer und der \gls{administrator} sind in einer Gruppe und beide angemeldet. In der Gruppe gibt es mindestens ein \gls{rezept}. Der Nutzer befindet sich in der Rezeptansicht des Rezepts.\\
\textbf{Nachbedingung Erfolg:} Das Rezept wird keinem Gruppenmitglied außer dem Autor angezeigt. Administratoren können das Rezept nur in der Gruppen-Detail-Ansicht sehen.\\
\textbf{Nachbedingung Fehlschlag:} -\\
\textbf{Akteure:} Nutzer, \gls{administrator}\\
\textbf{Auslösendes Ereignis:} -\\
\textbf{Beschreibung:}
\begin{enumerate}
    \item Der Nutzer klickt auf den Melden-Button
    \item Ein Pop-Up-Fenster erscheint, welches um Bestätigung bittet, dass das Rezept wirklich gemeldet werden soll.
    \item Der Nutzer klickt auf den Bestätigen-Button.
    \item Der \gls{administrator} erhält eine E-Mail mit der Benachrichtigung über die Meldung des Rezepts.
    \item Der \gls{administrator} öffnet die App und manövriert zur Gruppen-Detail-Ansicht.
    \item Der \gls{administrator} klickt in der Gruppen-Detail-Ansicht auf den Ausblenden-Button des Rezepts.
    \item Das Rezept wird ausgeblendet und ist nur noch für den Autor sichtbar. Für Administratoren ist es nur in der Gruppen-Detail-Ansicht sichtbar.
\end{enumerate}
\textbf{Alternativen:} Der Administrator kann das Rezept auch wieder einblenden, indem er auf den Einblenden-Button klickt.
\newpage

\subsection{Gruppennamen ändern}
\textbf{Anwendungsfall:} Ein \gls{administrator} möchte den Gruppennamen ändern.\\
\textbf{Anforderungen:} \ref{rc:GroupRenaming}\\
\textbf{Ziel:} Ein \gls{administrator} kann den Gruppennamen nach seinen Vorstellungen anpassen.\\
\textbf{Vorbedingung:} Der Nutzer ist \gls{administrator} der Gruppe, angemeldet und in der Gruppen-Detail-Ansicht.\\
\textbf{Nachbedingung Erfolg:} Der angepasste Gruppenname wird den Mitgliedern angezeigt.\\
\textbf{Nachbedingung Fehlschlag:} Der Gruppenname wurde nicht geändert.\\
\textbf{Akteure:} \gls{administrator}\\
\textbf{Auslösendes Ereignis:} In der Gruppen-Detail-Ansicht wird auf den Namen-Bearbeiten-Button geklickt.\\
\textbf{Beschreibung:}
\begin{enumerate}
    \item Ein Pop-Up-Fenster erscheint, welches um den neuen Gruppennamen bittet.
    \item Der Nutzer gibt den neuen Gruppennamen ein.
    \item Der Nutzer klickt auf den Bestätigen-Button.
    \item Der Nutzer wird zur Gruppen-Detail-Ansicht weitergeleitet, wo der neue Name angezeigt wird.
\end{enumerate}
\textbf{Alternativen:} Wird das Pop-Up-Fenster geschlossen, so wird die Benennung abgebrochen.
\newpage

\subsection{Administrator verlässt die Gruppe}
\textbf{Anwendungsfall:} Der einzige \gls{administrator} verlässt die Gruppe.\\
\textbf{Anforderungen:} \ref{rs:NoAdmins}\\
\textbf{Ziel:} Ein \gls{administrator} kann die Gruppe verlassen und ein anderes Mitglied wird zum neuen \gls{administrator}.\\
\textbf{Vorbedingung:} Der Nutzer ist einziger \gls{administrator} der Gruppe und die Gruppe enthält mehr als ein Mitglied. Zudem ist der Administrator angemeldet und befindet sich in der Gruppen-Detail-Ansicht.\\
\textbf{Nachbedingung Erfolg:} Der Nutzer der am längsten in der Gruppe ist, exklusive des bisherigen \glslink{administrator}{Administrators}, ist neuer \gls{administrator}.\\
\textbf{Nachbedingung Fehlschlag:} -\\
\textbf{Akteure:} Nutzer, \gls{administrator}\\
\textbf{Auslösendes Ereignis:} Der Administrator drückt auf den Gruppe-Verlassen-Button.\\
\textbf{Beschreibung:}
\begin{enumerate}
    \item Ein Pop-Up-Fenster erscheint, welches um Bestätigung bitten.
    \item Der \gls{administrator} klickt auf den Button zum Bestätigen.
    \item Der \gls{administrator} wird zur Verwaltungsansicht weitergeleitet.
    \item Der Nutzer der am längsten in der Gruppe ist, exklusive des bisherigen \glslink{administrator}{Administrators}, ist neuer \gls{administrator}. 
\end{enumerate}
\textbf{Alternativen:} Ist der \gls{administrator} das letzte Mitglied der Gruppe, so wird die Gruppe gelöscht.
\resetsubsectionnumbering
\newpage

\section{Produktdaten}
\enablesubsectionnumbering{D}
Um die App nutzen zu können, ist es erforderlich einige Daten zu speichern. Diese werden auf unterschiedlichen Geräten gespeichert:

\subsection{Smartphone-Daten}
\begin{itemize}
    \item Anwendung
    \item Zwischengespeicherte Rezepte
\end{itemize}

\subsection{Server-Daten}
\begin{itemize}
    \item Rezepte
    \item Nutzerdaten
    \item Gruppen
\end{itemize}

\resetsubsectionnumbering
\section{Nichtfunktionale Anforderungen}
Im folgenden Kapitel werden die nichtfunktionalen Anforderungen und Qualitätsmerkmale der App definiert.
Anschließend werden die wichtigsten Qualitätsmerkmale operationalisiert und, falls diese nicht als allgemeine Richtlinie (z.B. Standard, Norm usw.) zu Verfügung gestellt werden,
als konkrete Produktanforderungen konkretisiert.

\subsection{Funktionalität}
\begin{tabular}{| c | c | c | c | c |}
    \hline
    \textbf{ Produktqualität } & \textbf{sehr gut} & \textbf{gut} & \textbf{normal} & \textbf{nicht relevant} \\ \hline
    Angemessenheit             & X                 &              &                 &                         \\ \hline
    Richtigkeit                &                   & X            &                 &                         \\ \hline
    Interoperabilität          & X                 &              &                 &                         \\ \hline
    Ordnungsmäßigkeit          &                   & X            &                 &                         \\ \hline
\end{tabular}

\textbf{Angemessenheit}\\
Da jede Komponente der App dem Nutzer entweder beim Teilen, Finden oder Speichern von \glslink{rezept}{Rezepten} unterstützt, ist die Angemessenheit der App insgesamt als sehr hoch einzustufen

\textbf{Richtigkeit und Ordnungsmäßigkeit}\\
Die App bietet nur an wenigen Schnittstellen wie z.B. dem Erstellen bzw. Verändern von \glslink{rezept}{Rezepten} Anfälligkeit für Verlust oder Verfälschung von Daten. Dennoch ist ein hohes Maß an Richtigkeit insbesondere im Bereich der Nutzer- und Rezeptverwaltung erwünscht.

\textbf{Interoperabilität}\\
Da die App mit Schnittstellen wie bspw. dem Server oder Android-Betriebssystem fehlerfrei kommunizieren muss, um korrekt zu arbeiten, ist eine sehr gute Interoperabilität wichtig.

\subsection{Sicherheit}
\begin{tabular}{| c | c | c | c | c |}
    \hline
    \textbf{Produktqualität} & \textbf{sehr gut} & \textbf{gut} & \textbf{normal} & \textbf{nicht relevant} \\ \hline
    Zuverlässigkeit          & X                 &              &                 &                         \\ \hline
    Reife                    & X                 &              &                 &                         \\ \hline
    Fehlertoleranz           &                   &              & X               &                         \\ \hline
\end{tabular}

\textbf{Zuverlässigkeit und Reife}\\
Durch das Durchführen von Tests während der Implementierung können Fehler früh gefunden werden und der fehlerhafte Code verbessert werden.
Durch die zahlreichen Testfälle können wir eine sehr gute Zuverlässigkeit und Reife der App gewährleisten.

\textbf{Fehlertoleranz}\\
Trotz kontinuierlichen Testens während des Entwicklungsprozesses, kann es zur Laufzeit des Spiels zu Fehlern kommen.
Das Ausmaß ist dabei von Fall zu Fall unterschiedlich.
Folglich wird die Fehlertoleranz der App als normal eingestuft.

\subsection{Benutzbarkeit}
\begin{tabular}{| c | c | c | c | c |}
    \hline
    \textbf{Produktqualität} & \textbf{sehr gut} & \textbf{gut} & \textbf{normal} & \textbf{nicht relevant} \\ \hline
    Verständlichkeit         & X                 &              &                 &                         \\ \hline
    Erlernbarkeit            & X                 &              &                 &                         \\ \hline
    Bedienbarkeit            &                   & X            &                 &                         \\ \hline
    Effizienz                &                   & X            &                 &                         \\ \hline
    Zeitverhalten            &                   & X            &                 &                         \\ \hline
    Verbrauchsverhalten      &                   & X            &                 &                         \\ \hline
\end{tabular}

\textbf{Verständlichkeit, Erlernbarkeit und Bedienbarkeit}\\
Die App soll intuitiv zugänglich sein, um ein hürdenfreie User Experience zu garantieren.

\textbf{Effizienz, Zeitverhalten und Verbraucherverhalten}\\
Die Effizienz der App muss als gut eingestuft werden, um ein möglichst langes Nutzererlebnis trotz des begrenzen Energiespeichers des Endgerätes zu realisieren.
Um dieses Ziel zu erreichen, soll die App ein relativ geringes Verbrauchs- und Zeitverhalten haben.


\subsection{Änderbarkeit}
\begin{tabular}{| c | c | c | c | c |}
    \hline
    \textbf{Produktqualität} & \textbf{sehr gut} & \textbf{gut} & \textbf{normal} & \textbf{nicht relevant} \\ \hline
    Analysierbarkeit         & X                 &              &                 &                         \\ \hline
    Modifizierbarkeit        & X                 &              &                 &                         \\ \hline
    Stabilität               &                   & X            &                 &                         \\ \hline
    Prüfbarkeit              & X                 &              &                 &                         \\ \hline
    Übertragbarkeit          &                   &              & X               &                         \\ \hline
    Anpassbarkeit            & X                 &              &                 &                         \\ \hline
    Installierbarkeit        &                   & X            &                 &                         \\ \hline
    Konformität              &                   &              & X               &                         \\ \hline
    Austauschbarkeit         & X                 &              &                 &                         \\ \hline
\end{tabular}

\textbf{Analysierbarkeit, Modifizierbarkeit, Prüfbarkeit, Anpassbarkeit und Austauschbarkeit} \newline
Analysierbarkeit, Modifizierbarkeit, Prüfbarkeit, Anpassbarkeit und Austauschbarkeit sind für späteres Erweitern und Verbessern der App unerlässlich und werden deshalb mit sehr gut eingestuft.

\textbf{Stabilität}\newline
Inhaltliche und designtechnische Änderungen der App dürfen keine Beeinträchtigung auf die Funktionalität der App haben.
Dadurch muss die Stabilität der App mit gut eingestuft werden.

\textbf{Installierbarkeit}\newline
Um die App spielen zu können, muss die App auf dem Android-Smartphone des Nutzers installiert werden.
Dadurch ist eine gute Installierbarkeit unerlässlich.

\textbf{Konformität}\newline
Die App basiert auf wenigen, klar definierten Schnittstellen. Das Verhalten und die Laufzeit auf einem Gerät beeinflusst das Verhalten nur geringfügig. Die Anforderung an die Konformität kann also als normal eingestuft werden

\subsection{Hardwareanforderungen}
Die App ist für Standardsmartphones, welche Android 8.0 oder höher unterstützen konzipiert. Zum Einloggen sowie zum Laden bzw. Hochladen neuer \glslink{rezept}{Rezepte} ist eine Internetverbindung erforderlich. Um Gruppen über QR-Codes beitreten zu können, wird außerdem eine Handykamera benötigt.


\subsection{Qualitätsanforderungen}
Die oben als am wichtigsten bezeichneten Qualitätsmerkmale werden im Folgenden operationalisiert, d.h. in konkreten Produktanforderungen konkretisiert oder es wird angegeben, welche Richtlinie (z. B. Standard, Norm) einzuhalten ist.

\begin{enumerate}[start=1,label={$\langle$\bfseries Q\arabic*$\rangle$}, leftmargin = 5em, itemsep=4pt, parsep=4pt]
    \item Passwörter sollen nur verschlüsselt gespeichert werden.
    \item Passwörter werden beim Zurücksetzen vom Nutzer ausgewählt, nicht automatisch generiert
    \item Der Code soll sinnvoll und hinreichend dokumentiert sein.
    \item Die App soll sinnvoll modularisiert sein.
    \item Die App soll auf Android-Version 8.0 und höher installierbar sein.
    \item Die Sprache der App soll deutsch sein.
\end{enumerate}

\section{Glossar}
\printglossary[style=altlist]
\end{document}