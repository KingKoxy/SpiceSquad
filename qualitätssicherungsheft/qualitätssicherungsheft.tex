\documentclass{qualitätssicherungsheft}
% glossary
\makeglossaries

\begin{document}
%Glossary entry
\newglossaryentry{squad}{
    name=Squad,
    description={Cooler Begriff für Gruppe}
}

\newglossaryentry{Admin}{
    name=Admin,
    description={Rolle, die innerhalb einer Gruppe eingenommen werden kann. Admins können Nutzer kicken, bannen, zu Admins befördern sowie Rezepte für Gruppenmitglieder unsichtbar machen. Der Ersteller einer Gruppe ist automatisch ein Admin. Admins können von anderen Admins zu normalen Nutzern degradiert werden.}
}

\newglossaryentry{privat}{
    name=privat,
    description={Ein privates Rezept ist nur für den Nutzer, der es erstellt hat, sichtbar.}
}

\newglossaryentry{Rezept}{
    name=Rezept,
    description={Ein Rezept besteht aus einer Liste an Zutaten mit Mengenangaben sowie einer Zubereitungsanweisung in Fließtext. Zusätzlich können Angaben zum Zeitaufwand und zur Schwierigkeit (Hinzufügedatum?) enthalten sein.}
}
\maketitle
\tableofcontents
\newpage

\section*{Gender-Hinweis}
Zur besseren Lesbarkeit wird in diesem Entwurfsheft das generische Maskulinum verwendet.
Die in diesem Heft verwendeten Personenbezeichnungen beziehen sich - sofern nicht anders kenntlich gemacht - auf alle Geschlechter.
\newpage
\section{Bugs}
\subsection{Keine Fehlermeldung beim Login mit falschem Passwort}
%TODO: Lukas muss genauen Fehlergrund und Behebung dokumentieren
\paragraph*{Beschreibung} Beim Login mit falschem Passwort wird keine Fehlermeldung mehr angezeigt. Dies war jedoch zu einem bestimmten Zeitpunkt noch nicht so.
\paragraph{Grund} Durch eine Änderung im Backend 
gibt eine fehlerhafte Anfrage den Code 500, statt wie zuvor den Code 401, zurück. Auf diesen Code wird jedoch nicht geprüft.
\paragraph{Behebung} Das Backend gibt nun wieder den Code 401 zurück.
\newpage
\subsection{Gruppenname kann nicht geändert werden}
\paragraph*{Beschreibung} Wird versucht den Gruppennamen zu ändern, so bleibt der alte Name bestehen.
\paragraph{Grund} Beim Konvertieren des neuen Gruppenobjektes zu einer JSON für die Anfrage an das Backend wird ein Fehler geworfen, da eine zusätzliche \texttt{.toMap()} Methode benötigt wird.
\paragraph{Behebung} Die \texttt{.toMap()} Methode wurde zur \texttt{Group}-Klasse im Frontend hinzugefügt.
\end{document}