\documentclass{qualitätssicherungsheft}
% glossary
\makeglossaries

\begin{document}
%Glossary entry
\newglossaryentry{squad}{
    name=Squad,
    description={Cooler Begriff für Gruppe}
}

\newglossaryentry{Admin}{
    name=Admin,
    description={Rolle, die innerhalb einer Gruppe eingenommen werden kann. Admins können Nutzer kicken, bannen, zu Admins befördern sowie Rezepte für Gruppenmitglieder unsichtbar machen. Der Ersteller einer Gruppe ist automatisch ein Admin. Admins können von anderen Admins zu normalen Nutzern degradiert werden.}
}

\newglossaryentry{privat}{
    name=privat,
    description={Ein privates Rezept ist nur für den Nutzer, der es erstellt hat, sichtbar.}
}

\newglossaryentry{Rezept}{
    name=Rezept,
    description={Ein Rezept besteht aus einer Liste an Zutaten mit Mengenangaben sowie einer Zubereitungsanweisung in Fließtext. Zusätzlich können Angaben zum Zeitaufwand und zur Schwierigkeit (Hinzufügedatum?) enthalten sein.}
}
\maketitle
\tableofcontents
\newpage

\section*{Gender-Hinweis}
Zur besseren Lesbarkeit wird in diesem Entwurfsheft das generische Maskulinum verwendet.
Die in diesem Heft verwendeten Personenbezeichnungen beziehen sich - sofern nicht anders kenntlich gemacht - auf alle Geschlechter.
\newpage

\section{Gefundene Fehler}
In diesem Abschnitt werden während der Qualitätssicherungsphase gefundene Fehler dokumentiert.

\subsection{Fehler 1}
\paragraph*{Beschreibung:} Server gibt bei Fehlern immer einen StatusCode 500 zurück trotz valider Durchführung der Anfrage, was zu fälschlicher Annahmen im Fontend führten (z.B. Pop-Up-Fenster wurden nicht angezeigt). 
\paragraph*{Entdeckt durch:} Integrationstests
\paragraph*{Fehlerbehebung:} Der Fehler wurde durch das Hinzufügen eines
next-Parameters in den Error-Handlern behoben. Dieser next-Parameter wurde fälschlicher weise entfernt, da er von der IDE als unnötig markiert wurde. 

\subsection{Fehler 2}
\paragraph*{Beschreibung:} Server wirft Fehler beim Erstellen von Gruppe und anschließendem Löschen einer Gruppe nach verlassen des letzten Nutzers.
\paragraph*{Entdeckt durch:} Unittest
\paragraph*{Fehlerbehebung:} Bei dem Fehler handelte es sich um einen logischen Fehler. Nach verlassen des letzten Nutzer wurde die Gruppe gelöscht und anschließend wurde nochmal versucht alle Nutzer der Gruppe zu finden, was einen leeres Array zurückgab. Dieses leere Array wurde dann versucht zu durchlaufen, was zu einem Fehler führte. Der Fehler wurde behoben, indem nach dem verlassen des letzten Nutzers die Gruppe gelöscht wird ohne weitere Operationen durchzuführen.

\section{Auswertung der Umfrageergebnisse}
In dieser Sektion werden die Umfrageergebnisse ausgewertet. Die Umfrage bestand aus vier Bewertungsfragen (Noten 1-6), drei Textfeldern zu Fehlerangabe und Verbesserungsvorschlägen sowie Angaben zu Alter und Geschlecht der befragten Person.

\subsection{Benutzbarkeit}
Während einzelne Nutzer Probleme bei der Bedienung der App hatten, bewerteten die meisten Nutzer die Benutzbarkeit positiv. Auch ältere Nutzer hatten meist keine Probleme, sich in der App zurechtzufinden. Die Durchschnittsnote war 1,8.

\subsection{Gestaltung}
Auch bei der Gestaltung war der Großteil der Nutzer zufrieden. Die Durchschnittsnote war 1,7.

\section{Tests im Backend}
Das Backend wurde mit Integrationstests getestet. Die Tests wurden mit dem Framework Mocha geschrieben. 
Die Tests sind in dem Ordner \textit{test} zu finden. Die Tests können mit dem Befehl \textit{npm test} ausgeführt werden.

\section{Nichtfunktionale Anforderungen}
\subsection{Änderbarkeit}
Die Änderbarkeit des Backends wurde durch die Verwendung von TypeScript und der Verwendung von REST-API sichergestellt.
Durch die Verwendung von Controllern mit dehr niedriger Kopplung ist eine Änderung oder Erweiterung der Funktionalität des Backends einfach möglich.

\subsection{Sicherheit}
Die Sicherheit des Backends wurde durch die Verwendung von Firebase sichergestellt. 
Dieses bietet als Google-Service eine sichere Nutzer- und Sessionverwaltung.

\section{Lasttests}
Da unsere Anwendung auf die Verwendung vieler Nutzer ausgelegt ist, haben wir Lastentests durchgeführt.
Diese wurden mit dem Tool Artillery durchgeführt.
Die Testumgebung sah dabei wie folgt aus:
Wir haben unsere Backend auf einem Server mit nahezu identischer Hardware wie der Server, auf dem die Anwendung später laufen soll, deployed.
Der Server besaß 2 vCores und 4 GB RAM.
Der Server der Anwendung besitzt 2 vCores und 2 GB RAM.
(Leider konnten wir im Rahmen des PSE keine bessere Testumgebung bereitstellen ohne auf monetär Mittel zurückzugreifen.)

Der Test bestand aus drei aufeinanderfolgenden Phasen.
In jeder Phase wurde die Anzahl der virtuellen Nutzer dabei Stufenweise im 10 Sekunden Intervall im Testzeitraum erhöht.
\begin{itemize}
    \item In der ersten Phase wurden für 60 Sekunden jede Sekunde 1 bis abschließend 5 virtuelle Nutzer erstellt.
    \item In der zweiten Phase wurden für 60 Sekunden jede Sekunde 5 bis abschließend 15 virtuelle Nutzer erstellt.
    \item In der letzten Phase wurden für 20 Sekunden jede Sekunde 15 bis abschließend 25 virtuelle Nutzer erstellt.
\end{itemize}

Die drei Phasen sollen dabei einmal ein leichtes bis normales, ein normales bis hohes und ein spitzen aufkommen von Anfragen simulieren.

Jeder Nutzer hatte im Test die selbe Aufgabe.
Er Nutzer sollte sich
\begin{itemize}
    \item einloggen
    \item alle Rezepte aufrufen
    \item einer Gruppe beitreten
    \item Namen von Zutaten aufrufen
    \item ein Rezept erstellen
    \item die Rezepte nach betrete der Gruppe erneut aufrufen
    \item und anschließend  und ein Bild aufrufen.
\end{itemize}
Mit diesem Ablauf haben wir versucht, die Anwendung mit einer realistischen Nutzung zu testen, die aus den Kernfunktionen der Anwendung besteht.
Dadurch, dass die Nutzer alle der gleichen Gruppe beitraten und ein Rezepte erstellten wurde das aufkommen der Anfragen über den Verlauf des Tests hinweg immer größer.
Damit konnten wir sicherstellen, dass die Anwendung auch bei großen Anfragen stabil bleibt.

\subsection{Bemerkung zur Aussagekraft der Ergebnisse}
Leider konnten wir unseren Test nur in einem limitierten Umfang durchführen.
Grund dafür war, dass unser Server leider zu wenig Rechenleistung besaß, um die Anfragen von vielen Nutzern zu verarbeiten.
Wir kamen so schon bei denn doch noch vergleichsweise geringen Nutzern in unserem Test an die komplette Auslastung des Servers und konnten so nicht exakt analysieren, wie unsere Anwendung sich bei einem hohen aufkommen von Anfragen verhält.
Deshalb sind die Ergebnisse des Tests nur bedingt aussagekräftig.

\subsection{Metriken}
Für die Auswertung des Tests haben wir uns auf zwei Metriken konzentriert.
Einmal die Median Antwortzeit und einmal den Apdex-Wert.
Die Median Antwortzeit gibt an, wie lange eine Anfrage im Median benötigt hat, um beantwortet zu werden.
Apdex (Application Performance Index) ist ein standardisiertes Maß zur Bewertung und Berichterstattung über die Zufriedenheit der Benutzer mit der Leistung von Anwendungen und Diensten. Es bietet einen einfachen, verständlichen Überblick über die Nutzerzufriedenheit und ist besonders nützlich für solche, die eine schnelle Einschätzung der Anwendungsperformance benötigen, ohne sich mit detaillierten Metriken auseinandersetzen zu müssen.

Der Apdex-Wert wird als Zahl zwischen 0 und 1 ausgedrückt, wobei:
\begin{itemize}
    \item 1 eine vollständige Zufriedenheit der Benutzer darstellt.
    \item 0 vollständige Unzufriedenheit der Benutzer bedeutet.
\end{itemize}

Dabei wird die Zeit in drei Bereiche eingeteilt.
Anfragen, die innerhalb einer bestimmten Zeit beantwortet wurden, werden als zufriedenstellend gewertet.
Anfragen, die länger als die bestimmte Zeit benötigen, werden als nicht zufriedenstellend gewertet.
Anfragen, die innerhalb der bestimmten Zeit beantwortet wurden, werden als tolerierbar gewertet.
Der Apdex-Wert ist dann der Anteil der zufriedenstellenden und tolerierbaren Anfragen an allen Anfragen.

Bei unserem Test haben wir uns für einen Schwellwert von 1.9 Sekunden entschieden. Wir haben uns für diesen Wert entschieden, da wir der Meinung sind, dass eine Anfrage, die länger als 1.9 Sekunden benötigt, nicht mehr zufriedenstellend ist. Hinsichtlich dem aufkommen von Daten, die wir bei Anfragen erwarten, ist dies in unseren Augen ein realistischer Wert.

\subsection{Ergebnisse}
In unserem Test erziehlten wir bei einem Apdex-Wert von 1.9s und einer Median Antwortzeit von 0.4879 Sekunden einen Score von 0.84. Damit wird die Anwendung als zufriedenstellend bewertet.

\begin{figure}[!htp]
    \centering
        \includegraphics[height=80mm]{images/lasttest/response_time.png}
        \caption[center]{HTTP Response Time}
        \label{fig:http_response_time}
\end{figure}

Wie man in der Abbildung \ref{fig:http_response_time} sehen kann, steigt die Antwortzeit mit der Anzahl der Nutzer an. Was zu erwarten war. Was jedoch auch auffällt, dass ab der dritten Phase die Antwortzeit über unsern Schwellwert von 1.9 Sekunden steigt. Dies ist darauf zurückzuführen, dass der Server nicht mehr in der Lage war, die Anfragen zu verarbeiten. Dies ist auch in der Abbildung \ref{fig:cpu_usage} zu sehen. Die CPU-Auslastung steigt mit der Anzahl der Nutzer an und erreicht schon am Ende der zweiten Phase 100\%. Dies ist auch der Grund, warum wir den Test nicht mit mehr Nutzern durchführen konnten und somit der Test, vor allem zum Ende hin, nur bedingt aussagekräftig wird.

In der letzten Phase ist zu sehen, dass die Antwortzeit weiter bis an die 10 Sekunden steigt. An diesem Zeitpunkt wird eine Anfrage als Timeout eingestuft. Der Test wies dabei 851 Timeouts auf. Das ist leider ein sehr hoher Wert. Dies ist jedoch aus unserer Sicht vor allem auf die geringe Rechenleistung des Servers zurückzuführen. 

Es kam beim gesammten Test nicht zum Absturz der Anwendung.
Jedoch war bei einem versuch mit höheren virtuellen Nutzerzahlen der Server nicht mehr in der Lage, die Anfragen zu verarbeiten und wurde vom Serverbetreiber aufgrund von Überlast neugestartet.

\begin{figure}
    \centering
        \includegraphics[height=30mm]{images/lasttest/cpu_usage.png}
        \caption[center]{CPU Usage}
        \label{fig:cpu_usage}
\end{figure}

\subsection{Fazit}
Der Test hat gezeigt, dass die Anwendung bei einer geringen Anzahl von Nutzern gut funktioniert. Jedoch wäre die Anwendung nach dem obigen Testergebnis nicht für eine große Anzahl von Nutzern ausgelegt. Dies ist aus unserer Sicht vor allem auf die geringe Rechenleistung des Servers zurückzuführen. Wir sind der Meinung, dass die Anwendung bei einer höheren Rechenleistung des Servers auch bei einer höheren Anzahl von Nutzern gut funktionieren würde.

Um diese Aussage zu bestätigen, müsste jedoch ein weiterer Test mit einer höheren Rechenleistung des Servers durchgeführt werden, was uns leider im Rahmen des PSE nicht möglich war.
\newpage

\section{User Storys aus dem Pflichtenheft}

Im Pflichtenheft haben wir insgesamt 14 Use-Case Beispiele detailliert beschrieben und diese jetzt alle ausgiebig getestet. Wir haben versucht alle Test so weit wie möglich zu automatisieren, sodass es möglich ist bei weiteren Releases der App das Funktionieren der bisherigen Funktionen sicherzustellen. Das Automatisieren wurde mithilfe des Flutter eigenen Packets Integration-Test umgesetzt und auf einem gesondertem Datensatz durchgeführt.

\subsection{⟨UC10⟩ Einloggen und Ausloggen}
Der Testfall konnte automatisiert werden und der Test wurde erfolgreich durchgeführt. 
\subsection{⟨UC20⟩ Registrierung und Konto löschen}
Der Testfall konnte automatisiert werden und der Test wurde erfolgreich durchgeführt.
\subsection{⟨UC30⟩ Passwort zurücksetzen}
Der Testfall konnte nicht automatisiert werden, da es keine Funktion gibt die auf einen Mailserver zugreifen kann innerhalb von Integration Tests. Das Use-Case Beispiel wurde per Hand mehrmals durchgeführt und alle Versuche waren erfolgreich.
\subsection{⟨UC40⟩ Benutzernamen und Profilbild ändern}
Der Testfall konnte größtenteils automatisiert werden. Das Ändern des Profilbilds war nicht möglich da es noch keine Methode zum übergeben von Dateien bei Integration Test gibt. Der Test wurde erfolgreich durchgeführt für den automatisierten Teil und auch für den Teil der manuell durchgeführt werden musste.
\subsection{⟨UC50⟩ Rezept erstellen, ändern und löschen}
Der Testfall konnte automatisiert werden und der Test wurde erfolgreich durchgeführt.
\subsection{⟨UC60⟩ Rezept suchen, sortieren, filtern, betrachten und favorisieren}
Der Testfall konnte automatisiert werden und der Test wurde erfolgreich durchgeführt. 
\subsection{⟨UC70⟩ Rezept auf privat stellen und exportieren}
Der Testfall konnte nicht automatisiert werden da keine Funktion gibt die Überprüft ob eine Datei heruntergeladen wurde. Das Use-Case Beispiel wurde per Hand mehrmals durchgeführt und alle Versuche waren erfolgreich.
\subsection{⟨UC80⟩ Gruppe erstellen, Gruppe mit Kürzel beitreten und Gruppe auflösen}
Der Testfall konnte automatisiert werden und der Test wurde erfolgreich durchgeführt. 
\subsection{⟨UC90⟩ Gruppe mit QR-Code beitreten und verlassen}
Der Testfall konnte nicht automatisiert getestet werden da keine Funktion gibt die auf eine Kamera zugreifen kann oder einen QR-Code erkennt. Das Use-Case Beispiel wurde per Hand mehrmals durchgeführt und alle Versuche waren erfolgreich.
\subsection{⟨UC100⟩ Administrator ernennen und Administratorenstatus entfernen}
Der Testfall konnte automatisiert werden und der Test wurde erfolgreich durchgeführt.
\subsection{⟨UC110⟩ Nutzer kicken und Nutzer bannen}
Der Testfall konnte automatisiert werden und der Test wurde erfolgreich durchgeführt.
\subsection{⟨UC120⟩ Rezepte verwalten für die Gruppen}
Der Testfall konnte nicht automatisch getestet werden, da kein Zugriff auf einen Mailserver möglich ist innerhalb von Integration Test um die Report-E-Mails zu lesen. Das Use-Case Beispiel wurde per Hand mehrmals durchgeführt und alle Versuche waren erfolgreich.
\subsection{⟨UC130⟩ Gruppennamen ändern}
Der Testfall konnte automatisiert werden und der Test wurde erfolgreich durchgeführt.
\subsection{⟨UC140⟩ Administrator verlässt die Gruppe}
Der Testfall konnte automatisiert werden und der Test wurde erfolgreich durchgeführt.

\end{document}
