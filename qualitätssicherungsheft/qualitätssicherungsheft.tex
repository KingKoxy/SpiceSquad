\documentclass{qualitätssicherungsheft}
% glossary
\makeglossaries

\begin{document}
%Glossary entry
\newglossaryentry{squad}{
    name=Squad,
    description={Cooler Begriff für Gruppe}
}

\newglossaryentry{Admin}{
    name=Admin,
    description={Rolle, die innerhalb einer Gruppe eingenommen werden kann. Admins können Nutzer kicken, bannen, zu Admins befördern sowie Rezepte für Gruppenmitglieder unsichtbar machen. Der Ersteller einer Gruppe ist automatisch ein Admin. Admins können von anderen Admins zu normalen Nutzern degradiert werden.}
}

\newglossaryentry{privat}{
    name=privat,
    description={Ein privates Rezept ist nur für den Nutzer, der es erstellt hat, sichtbar.}
}

\newglossaryentry{Rezept}{
    name=Rezept,
    description={Ein Rezept besteht aus einer Liste an Zutaten mit Mengenangaben sowie einer Zubereitungsanweisung in Fließtext. Zusätzlich können Angaben zum Zeitaufwand und zur Schwierigkeit (Hinzufügedatum?) enthalten sein.}
}
\maketitle
\tableofcontents
\newpage

\section*{Gender-Hinweis}
Zur besseren Lesbarkeit wird in diesem Entwurfsheft das generische Maskulinum verwendet.
Die in diesem Heft verwendeten Personenbezeichnungen beziehen sich - sofern nicht anders kenntlich gemacht - auf alle Geschlechter.
\newpage

\section{Gefundene Fehler}
In diesem Abschnitt werden während der Qualitätssicherungsphase gefundene Fehler dokumentiert.

\subsection{Fehler 1}
Beschreibung: Server gibt bei Fehler immer 500 zurück. \newline
Entdeckt durch: Integrationstests \newline
Fix: Fehler in Errorhandler behoben.

\subsection{Fehler 2}
Beschreibung: Server gibt bei Erstellen von Gruppe und anschließendem Löschen anschließend an jedem Endpunkt Fehler 502 aus \newline
Entdeckt durch: Tester Henri Becker \newline
Fix: 

\section{Tests im Backend}
Das Backend wurde mit Integrationstests getestet. Die Tests wurden mit dem Framework Mocha geschrieben. 
Die Tests sind in dem Ordner \textit{test} zu finden. Die Tests können mit dem Befehl \textit{npm test} ausgeführt werden.

\section{Nichtfunktionale Anforderungen}
\subsection{Änderbarkeit}
Die Änderbarkeit des Backends wurde durch die Verwendung von TypeScript und der Verwendung von REST-API sichergestellt.
Durch die Verwendung von Controllern mit dehr niedriger Kopplung ist eine Änderung oder Erweiterung der Funktionalität des Backends einfach möglich.

\subsection{Sicherheit}
Die Sicherheit des Backends wurde durch die Verwendung von Firebase sichergestellt. 
Dieses bietet als Google-Service eine sichere Nutzer- und Sessionverwaltung.




\end{document}