\documentclass{qualitätssicherungsheft}
% glossary
\makeglossaries

\begin{document}
\newglossaryentry{label}{
    name=Label,
    plural=Labels,
    description={Rezepte können mit bezeichnenden Stichwörtern, sogenannten Labels, versehen werden. Dies ermöglicht das Filtern von Rezepten nach bestimmten Eigenschaften (z.B. vegetarisch, glutenfrei, halal)}
}
\maketitle
\tableofcontents
\newpage

\section*{Gender-Hinweis}
Zur besseren Lesbarkeit wird in diesem Entwurfsheft das generische Maskulinum verwendet.
Die in diesem Heft verwendeten Personenbezeichnungen beziehen sich - sofern nicht anders kenntlich gemacht - auf alle Geschlechter.
\newpage
\section{Bugs}
\subsection{Keine Fehlermeldung beim Login mit falschem Passwort}
%TODO: Lukas muss genauen Fehlergrund und Behebung dokumentieren
\paragraph*{Beschreibung} Beim Login mit falschem Passwort wird keine Fehlermeldung mehr angezeigt. Dies war jedoch zu einem bestimmten Zeitpunkt noch nicht so.
\paragraph{Grund} Durch eine Änderung im Backend
gibt eine fehlerhafte Anfrage den Code 500, statt wie zuvor den Code 401, zurück. Auf diesen Code wird jedoch nicht geprüft.
\paragraph{Behebung} Das Backend gibt nun wieder den Code 401 zurück.
\newpage
\subsection{Gruppenname kann nicht geändert werden}
\paragraph*{Beschreibung} Wird versucht den Gruppennamen zu ändern, so bleibt der alte Name bestehen.
\paragraph{Grund} Beim Konvertieren des neuen Gruppenobjektes zu einer JSON für die Anfrage an das Backend wird ein Fehler geworfen, da eine zusätzliche \texttt{.toMap()} Methode benötigt wird.
\paragraph{Behebung} Die \texttt{.toMap()} Methode wurde zur \texttt{Group}-Klasse im Frontend hinzugefügt.
\newpage
\subsection{Fehlende Übersetzung der Konto-Löschen-Dialoge}
\paragraph*{Beschreibung} Die Dialoge zum Löschen eines Kontos sind nicht auf Englisch übersetzt.
\paragraph{Grund} Der Text ist als Magic-String im Code hartkodiert.
\paragraph{Behebung} Die Übersetzungen wurden, wie die anderen Texte, in die l10n-Dateien ausgelagert.
\newpage
\subsection{Halale Rezepte werden auch als koscher angezeigt}
\paragraph*{Beschreibung} Rezepte, die halal sind, werden in der Rezept-Detail-Ansicht als koscher angezeigt, sonst jedoch nirgends.
\paragraph{Grund} In der Liste der Labels wird überprüft, ob ein bestimmtes Label aktiv ist und zeigt gegebenen falls das passende Widget an. Bei der If-Anfrage zum Koscher-Label ist dabei ein Fehler unterlaufen und es wurde auf Halal geprüft.
\paragraph{Behebung} Die If-Anfrage wurde korrigiert.
\newpage
\subsection{Keine Weiterleitung beim Speichern von Änderungen eines Rezepts}
\paragraph*{Beschreibung} Drückt man beim Bearbeiten eines Rezepts auf den Zurück-Knopf, so wird ein Dialog angezeigt, der fragt ob abgebrochen, verworfen oder gespeichert werden soll. Drückt man auf Speichern, so werden die Änderungen zwar gespeichert, der Nutzer aber nicht weitergeleitet.
\paragraph{Grund} Es wurde nur ein \texttt{Navigator.pop()} aufgerufen, welches den Dialog schließt. Es müsste aber noch ein weiteres \texttt{Navigator.pop()} aufgerufen werden, um den Nutzer zurück vorherigen Seite zu leiten.
\paragraph{Behebung} Es wird nun ein weiteres \texttt{Navigator.pop()} aufgerufen.
\newpage
\end{document}