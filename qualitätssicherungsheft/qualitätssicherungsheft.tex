\documentclass{qualitätssicherungsheft}
% glossary
\makeglossaries

\begin{document}
\newglossaryentry{label}{
    name=Label,
    plural=Labels,
    description={Rezepte können mit bezeichnenden Stichwörtern, sogenannten Labels, versehen werden. Dies ermöglicht das Filtern von Rezepten nach bestimmten Eigenschaften (z.B. vegetarisch, glutenfrei, halal)}
}
\maketitle
\tableofcontents
\newpage

\section*{Gender-Hinweis}
Zur besseren Lesbarkeit wird in diesem Entwurfsheft das generische Maskulinum verwendet.
Die in diesem Heft verwendeten Personenbezeichnungen beziehen sich - sofern nicht anders kenntlich gemacht - auf alle Geschlechter.
\newpage

\section{Gefundene Fehler}
In diesem Abschnitt werden während der Qualitätssicherungsphase gefundene Fehler dokumentiert.

\subsection{Fehler 1}
Beschreibung: Server gibt bei Fehler immer 500 zurück. \newline
Entdeckt durch: Integrationstests \newline
Fix: Fehler in Errorhandler behoben.

\subsection{Fehler 2}
Beschreibung: Server gibt bei Erstellen von Gruppe und anschließendem Löschen anschließend an jedem Endpunkt Fehler 502 aus \newline
Entdeckt durch: Tester Henri Becker \newline
Fix: 

\section{Tests im Backend}
Das Backend wurde mit Integrationstests getestet. Die Tests wurden mit dem Framework Mocha geschrieben. 
Die Tests sind in dem Ordner \textit{test} zu finden. Die Tests können mit dem Befehl \textit{npm test} ausgeführt werden.

\section{Nichtfunktionale Anforderungen}
\subsection{Änderbarkeit}
Die Änderbarkeit des Backends wurde durch die Verwendung von TypeScript und der Verwendung von REST-API sichergestellt.
Durch die Verwendung von Controllern mit dehr niedriger Kopplung ist eine Änderung oder Erweiterung der Funktionalität des Backends einfach möglich.

\subsection{Sicherheit}
Die Sicherheit des Backends wurde durch die Verwendung von Firebase sichergestellt. 
Dieses bietet als Google-Service eine sichere Nutzer- und Sessionverwaltung.




\end{document}